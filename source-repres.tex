% 
%  thoughtsnotthought.tex
%  
%  Created by Maarten Steenhagen on 2013-07-01.
%  Copyright 2013 Maarten Steenhagen. All rights reserved.
% 
\documentclass[sloppy, journal, git, bytitle, dodraft]{humapap}
\usepackage{soul}
\myauthor{Maarten Steenhagen}
\myemail{\\m.steenhagen.09@ucl.ac.uk\\} % Will be used on first page, loaded in author.tex
\mydraft{Draft version}
\theaffiliation{}
\thetitle {source representationalism} % Fist page (no caps is fine)
%\mysubtitle{a second title} % First page
%\mydraft{Draft - do not cite}
\thehtitle{the header} % Title in the header (no caps is fine)
\mydescription{Draft version} % Hovers over the title. Can be left blank (~)
\thejournal{the journal} % Left corner subsequent pages
\theyear{2014}
\thanks{}
\remote{https://bitbucket.org/msteenhagen/humapap/commits/}
\begin{document}
\documenttitle

% \begin{abstract}That some objects are, as the naive realist has it, presented in perception does not rule out that some objects of perception are represented. I show that at least some plausible versions of naive realism to accept that audition essentially tends towards representational perception. The argument for this is that,  on the one hand, our capacity for auditory perception is for the sake of hearing the sources of sounds, while on the other those sources, if they are not sounds, cannot be constituents of auditory perception. This implies that audition essentially has a tendency to afford perception of objects that need not be constituents of experience, which in turns suggests that such perceptions defy the non-representationalist analysis offered by the naive realist and, hence, must be represented in experience.\end{abstract}
	
	%Don't forget: represented = perceived & not presented. (Instead of presence in absence, it is perception in absence.)

% INTRODUCTION (fold)
% Make sure this paper is maximally concrete!
\dropcap{S}{ource representationalism} is a view about auditory perception, according to which the sources of the sounds humans hear are always represented in their perceptual experience. When we listen to and hear a piano, our auditory experience may consist in a perceptual relation to its sound, it can not consist in a perceptual relation to the piano itself. In this paper I will defend source representationalism on the basis of some considerations about listening to recordings of everyday sounds. 

% Introduce key example. Argue: We hear the sound of the piano when we listen to the recording. Either the same sound or a similar sound - this depends on one's view of recording and playback of sound. 
\sect Consider a situation in which you decide to listen to a stereo recording of Tatiana Nikolayeva's rendition of \emph{Die Kunst der Fuge}, a famous piano performance of J. S. Bach's baroque masterpiece that took place on a winter evening in 1953. In contrast to its original audience which found itself in the massive Building in City, you are currently seated in London, in a small and quiet room with only some books, a writing desk, and a high fidelity speaker set in your vicinity. You know that both the quality of the recording and the calibration of the loudspeakers in the room are superb, and sit back at ease, eyes closed, well aware that you're in for a treat. Carefully you press the button that will start playback. What do you hear? 

First of all, you hear the sounds of a piano. These particular sounds are perceived as soon as they fill the room when the recording starts to play back. Some will be loud, while others are muted; some may be high-pitched while others of lower audible frequencies. But is is undeniable that you can hear these sounds.   

% Introduces the hallucinatory case used later
Contrast this with the use of a device that, instead of playing back sounds, causes us to have auditory hallucinations of the sounds of a piano. Such a device, surely conceivable, would cause auditory sensations that in some respects resemble tinnitus, a sensation of ringing in the ears. A playback on such a device would result your having a series of auditory sensations that resembles hearing a performance of, say, Bach's \emph{Kunst der Fuge}. What is interesting about this device is that `listening' to music played back on it would not involve hearing any sound. Yet, it should be clear that listening to the ordinary playback of the Bach recording does not involve hallucinating sound. 

% Spells out qualms about reproducability: takes a neutral stance. (But keep an eye on this!) The point I need to hold on to is that we can hear the sound of a piano, and so the resemblance view must be developed as giving a gloss of that phrase as well. (Perhaps: what matters is that we hear a particular, whether abstract or concrete. The particular determines character. This particular could occur both in recording and actual cases.)
%We can use the idea of a prorietary sound here: a badly oiled revolving door may make the sound of a dying animal; sounds like a dying animal. The speaker sounds like a piano: this would not be a correct description. We do not hear the speaker.
Some have suggested that it is of the nature of sounds that they are reproducible. If they are right, then your speakers would re-produce the particular sounds that could be heard by the audience on the occasion of the actual performance as well. In other words, in hearing the tones you hear when listening to the recording, you may be hearing the same sounds that were produced by the piano Nikolayeva played on the evening of the concert. I believe this is a plausible way of capturing the nature of audio-reproduction. 

For whatever reason, some may wish to resist considering sounds to be abstract in this way. Yet even they can and should agree that in listening to the playback of a recording of the piano concert one genuinely hears sounds, sounds that on a well-attuned stereo set resemble in most respects the sounds that filled the concert hall on the evening of the performance. On such a picture, when we say that we hear the sounds of a piano, what we must mean is that we hear sounds that strikingly resemble the sounds of a piano.

% Argue: We can assume that we can hear the sound of a piano. We hear the piano when we listen to the recording. Two arguments: 
% [1] Generalisation argument: (softens up)
%If we can hear a piano as audience, then we we can hear a piano as listeners to recording; we can hear a piano as audience; we can hear a piano as listeners to recording. 
%[2] Phenomenological argument: (establishes)
%What will I do with the observation that we do not hear the speaker when we hear the sounds it reproduces. (We can distinguish between a speaker's producing and reproducing a sound, just as we can distinguish between someone saying something and someone rehearsing something.)
\sect I want to move on by arguing that we hear Nikolayeva's piano as well. For this I have two arguments. The first makes use of a conditional, of which I assume the antecedent is true. If an audience at Nikolayeva's concert was able to hear her piano when they heard the sounds that filled the room that evening, then an audience listening to the recording of those sounds will be able to her piano when they hear the played-back sounds that fill their room on that occasion. % It may be that this argument is already too similar to what I want to develop later on. It would be good if I have a strong argument from phenomenology. I could use Heidegger? 

% I should take the 'only if there's a continuous interaction' reaction seriously.


The second argument has to do simply with our experience of listening to a recording. The argument makes use of the observation that we can `listen our for' some particular object in hearing a series of sounds. [Example]

If we can `hear out for' a piano, then we can hear a piano; If we can `hear out for' a piano in some context, then we can hear a piano in that context; being a listener to recording is being in some context; If we can `hear out for' a piano as listeners to recording, then we can hear a piano as listeners to recording; we can `hear out for' a piano as listeners to recording; we can hear a piano as listeners to recording.


Simeon ten Holt's \emph{Incantation IV} is a modestly minimalist composition for four pianos. Imagine one is an apprentice piano tuner and has been given the job of tuning only one of the piano's used in a concert that will be recorded for a CD release. After the fact, one can listen to the recording and listen out for the piano one has tuned. Here one isn't interested in hearing  
 Imagine that one has tuned the piano strings that are struck by one of the hammers associated with a key on the piano. One can now listen to the recording of the concert and listen out for that particular piano one tuned. I believe that engaging in this activity successfully is certainly conceivable, and hence that one must be able to hear a piano when one listens to a recording. 
 
 
Some may want to resist this on the basis of a conviction that not even the audience during the concert was able to hear the piano. They could either say that this is because all we ever hear are sounds, and piano is no sound. Or they could say, weaker, that all we ever hear are events, and a piano is no event. I will set aside the former view and simply assume that it is false that all we ever hear are sounds. To the latter view, I want to suggest that even if it is true that one could only hear events. the argument I develop can be reformulated without difficulty: wherever I speak of hearing a piano being played, these people are invited to read `hearing a piano's being played'.
% These are older elaborations of the above point
%As I said, I will simply assume that we hear a piano when we go to a concert hall and listen to a performance of a piano composition played on a piano. Those who disagree will be those who think that sources of sounds are never the objects of auditory perception. I will assume they are wrong. 
%Some might object and claim that they admit that sources of sounds are objects of perception, but that these sources are not object simpliciter, but always objects as and when they are involved in some event of sounding or noisy change. They will emphasise that we can only hear an anvil \emph{being struck}, Big Ben \emph{striking}, or a piano \emph{being played}. I think this is correct, but it would present a specification and not a denial of the claim I want to make. This is because hearing a piano being played is just one kind of hearing a piano. And so if an audience could only hear the piano being played on the evening of Nikolayeva's performance, they would still be able to hear a piano. 

% Argue: If we hear the piano when we listen to the recording, then we perceive a piano that is not present to us. If we perceive something not present to us, the thing is represented in experience. If we hear the piano when we listen to the recording, then the piano is represented in experience when we listen to the recording. We hear the piano when we listen to the recording. The piano is represented in experience when we listen to the recording.
\sect


% Argue: If the piano is represented in experience when we listen to the recording, then the piano is represented in experience when we listen to it as an audience. [[WORK: This is because these experiences are subjectively indistinguishable, and belong to the same perceptual kind. ]]
\sect In this section I generalise the previous observation to all cases of sources representationalism. This turns the previous conclusion that some source hearing is representational into a stronger thesis: all source hearing is representational. I want to make use of a modified version of the argument from hallucination. 

This is a common formulation of the argument from hallucination (I follow Crane here): 

FIRST: It seems possible for someone to have an experience—a hallucination—which is subjectively indistinguishable from a genuine perception but where there is no mind-independent object being perceived.

This premise can be reformulated to be about the piano. It is possible for someone to have an experience which is subjectively indistinguishable from listening to a piano that is presently played in one's vicinity. 

SECOND: The perception and the subjectively indistinguishable hallucination are experiences of essentially the same kind.

This premise can be reformulated as a claim about the experience of the person listening to a recording on the one hand, and the experience of the person listening to the piano actually being played on the other. The claim would be that these are experiences of essentially the same kind. There will be controversy here about how to understand `essentially'. Perhaps I should rephrase this in terms of `fundamentally' and stick to Martin's formulation, as my argument is in part addressed to him (though only obliquely).

THIRD: Therefore it cannot be that the essence of the perception depends on the objects being experienced, since essentially the same kind of experience can occur in the absence of the objects.

Here we need to do a lot of interpretation. The reformulation needs to be that it cannot be that the essence of the experience of a member of the audience depends on the piano's being present, since essentially the same kind of experience can occur in the absence of the piano. Again, the issue will be about spelling out what we mean by `same kind' here.

CONCLUSION: Therefore the ordinary conception of perceptual experience—which treats experience as dependent on the mind-independent objects around us—cannot be correct.

I want to reach a different conclusion of course. I want to argue that the ordinary perceptual experience of hearing a piano does not depend on the presence of the piano (but only on the presence of a sound of a piano). 

It will take me too long to comment on the argument from hallucination in such detail, and so I should first run my own version and only afterwards mention that it in fact is an application of the argument from hallucination. This will be fairly obvious anyway.

There is something funny going on, and I want to pause and describe this. We can make two moves: 
When we hear a sound and its source, then an experience of that kind will be subjectively indistinguishable from hearing  a recording of that sound. 
When we hear a sound and its source, then an experience of that kind will be subjectively indistinguishable from hearing a the source actually produce the sound.
I recall from discussions with Roberta that there is an asymmetry that the naive realist exploits in resisting the argument from hallucination. This has something to do with phenomenal character. That might be a way in to make the argument work for source representationalism. 

Someone who disagrees with me on this point, what will they think? They will agree that hearing the piano when we listen to a recording is representational. But they will claim that when the piano is played to us actually, we will be able to hear the piano in a non-representational way. So, by hypothesis the piano is present and we hear the same (or a similar) sound. The question is, does this add something to the phenomenology of the experience? 

Reflect on the classical argument again. Here, the hallucination and the perception are said to have the same phenomenology. Yet, this is only true if we assume that there is a common element involved in both. Yet, someone might deny that there is a common element involved, because someone may deny that there is anything we become aware of in a hallucination. A hallucination may be defined just as an experience that is subjectively indistinguishable from a veridical perception. 

The sound case is interesting, because there we have a common element of more or less the kind the sense datum theorist at least dreamt of. Should this be my main focus? Should I claim that the `common kind' assumption must be made for the comparison I am interested in? This leads to three questions. 

1. Can sounds be a common factor of the relevant sort?
2. Does assuming the common factor claim validate the argument from hallucination? 
3. Does assuming the common factor claim validate the argument for source representationalism? 
 

% The experience of tinnitus does not similarly move us to accept that hearing sounds is representational. But we have seen that we should accept source representationalism: that hearing sources of sounds is representational. 
\sect



I want to defend the following conditional: if one accepts that in listening to a recording to a piano one hears a piano, then one must accept that the original audience’s auditory perception of the piano was representational. This is because in listening to a recording of the concert one would be able to hear an instrument that is absent from one’s current perceptual situation. Given that the perceptual experience of listening to a recording of a piano and listening to a real piano can be subjectively indistinguishable, we must accept that they are experiences of the same kind. Given that we have good reason to think that listening to a recording involves perceptual representation, we should also accept that listening to the actual piano involves perceptual representation. 

Here is the main argument again. `(C1)' means that these are conclusions of further arguments I need to offer.

\begin{itemize}
	\item We hear the sound of the piano when we listen to a recording of a piano being played (C1)
	\item We hear a piano when we listen to a recording of a piano being played (C2)
	\item If we hear a piano when we hear€ a recording of a piano being played, then a piano is represented in perception (C3)
	\item If piano is represented in perception, then the perception of a piano does not depend on the presence of a piano (C4) 
	\item If a perception of a piano does not depend on the presence of a piano, then a piano is represented in experience (C5)
	\item If we hear an actual piano being played we have an experience of fundamentally the same kind as listening to a recording of a piano being played (C6)
	\item If we hear an actual piano being played, then the perception of a piano does not depend on the presence of a piano (C7: From C4 and C6)
	\item If we hear an actual piano being played, then a piano is represented in experience (C8: From C5 and C7)
\end{itemize}

%Bibliography: \standardbib just loads a regular bibliography. All files have been loaded in the preamble.
\printbibliography
\end{document}

