% 
%  thoughtsnotthought.tex
%  
%  Created by Maarten Steenhagen on 2013-07-01.
%  Copyright 2013 Maarten Steenhagen. All rights reserved.
% 
\documentclass[sloppy, journal, git, bytitle, dodraft]{humapap}
\usepackage{soul}
\myauthor{Maarten Steenhagen}
\myemail{\\m.steenhagen.09@ucl.ac.uk\\} % Will be used on first page, loaded in author.tex
\mydraft{Draft version}
\theaffiliation{}
\thetitle {source representationalism} % Fist page (no caps is fine)
%\mysubtitle{a second title} % First page
%\mydraft{Draft - do not cite}
\thehtitle{the header} % Title in the header (no caps is fine)
\mydescription{Draft version} % Hovers over the title. Can be left blank (~)
\thejournal{the journal} % Left corner subsequent pages
\theyear{2014}
\thanks{}
\remote{https://bitbucket.org/msteenhagen/humapap/commits/}
\begin{document}
\documenttitle

% \begin{abstract}That some objects are, as the naive realist has it, presented in perception does not rule out that some objects of perception are represented. I show that at least some plausible versions of naive realism to accept that audition essentially tends towards representational perception. The argument for this is that,  on the one hand, our capacity for auditory perception is for the sake of hearing the sources of sounds, while on the other those sources, if they are not sounds, cannot be constituents of auditory perception. This implies that audition essentially has a tendency to afford perception of objects that need not be constituents of experience, which in turns suggests that such perceptions defy the non-representationalist analysis offered by the naive realist and, hence, must be represented in experience.\end{abstract}
	
	%Don't forget: represented = perceived & not presented. (Instead of presence in absence, it is perception in absence.)
	
	
	

% INTRODUCTION (fold)
% Make sure this paper is maximally concrete!
\dropcap{S}{ource representationalism} is a view about auditory perception, according to which the sources of the sounds humans hear are always represented in their perceptual experience. When we listen to and hear a piano, our auditory experience may consist in a perceptual relation to its sound, it can not consist in a perceptual relation to the piano itself. In this paper I will defend source representationalism on the basis of some considerations about listening to recordings of everyday sounds. 
% INTRODUCTION (end)

% SOUND (fold)
% Introduce key example. Argue: We hear the sound of the piano when we listen to the recording. Either the same sound or a similar sound - this depends on one's view of recording and playback of sound. 
\sect Consider a situation in which you decide to listen to a stereo recording of Tatiana Nikolayeva's rendition of \emph{Die Kunst der Fuge}, a famous piano performance of J. S. Bach's baroque masterpiece that took place on a winter evening in 1953. In contrast to its original audience which found itself in the massive Building in City, you are currently seated in London, in a small and quiet room with only some books, a writing desk, and a high fidelity speaker set in your vicinity. You know that both the quality of the recording and the calibration of the loudspeakers in the room are superb, and sit back at ease, eyes closed, well aware that you're in for a treat. Carefully you press the button that will start playback. What do you hear? 

% You hear the sound of a piano
First of all, you hear the sounds of a piano. These particular sounds are perceived as soon as they fill the room when the recording starts to play back. Some will be loud, while others are muted; some may be high-pitched while others of lower audible frequencies. But is is undeniable that you can hear these sounds.   

% Introduces the hallucinatory case used later
Contrast this with the use of a device that, instead of playing back sounds, causes us to have auditory hallucinations of the sounds of a piano. A playback on such a device would result your having a series of auditory sensations, sensations a bit like tinnitus but a lot more complex, and that resemble hearing a performance of, say, Bach's \emph{Kunst der Fuge}. `Listening' to Bach's music here would not involve hearing any sound. Yet, it should be clear that ordinary experience of playing back of sound recordings does not depend on hallucinating sound. 

% Spells out qualms about reproducability: takes a neutral stance. (But keep an eye on this!) The point I need to hold on to is that we can hear the sound of a piano, and so the resemblance view must be developed as giving a gloss of that phrase as well. (Perhaps: what matters is that we hear a particular, whether abstract or concrete. The particular determines character. This particular could occur both in recording and actual cases.)
%We can use the idea of a prorietary sound here: a badly oiled revolving door may make the sound of a dying animal; sounds like a dying animal. The speaker sounds like a piano: this would not be a correct description. We do not hear the speaker.
If we say that, seated in our room, we hear the sound of a piano, what does this mean? Some have suggested that it is of the nature of sounds that they are reproducible. If they are right, then your speakers would \emph{re}produce the particular sounds that could be heard by the audience on the occasion of the actual performance as well. In other words, in hearing the tones reproduced by your loudspeakers, you may be hearing the same sounds that were produced by the piano Nikolayeva played on the evening of the concert. For whatever reason, however, some may wish to resist considering sounds to be reproducible---to be \emph{abstract} in this way. Nonetheless, even they can agree that in listening to the playback one hears some particulars sounds that, given a well-attuned stereo set, are likely to be heard as the sounds of a piano. %Needs work 

%OLD: resemble strikingly sounds that filled the concert hall on the evening of the performance. On such a picture, when we say that we hear the sounds of a piano, what we must mean is that we hear sounds that strikingly resemble the sounds of a piano.

% Argue: We can assume that we can hear the sound of a piano. We hear the piano when we listen to the recording. Two arguments: 
% [1] Generalisation argument: (softens up)
%If we can hear a piano as audience, then we we can hear a piano as listeners to recording; we can hear a piano as audience; we can hear a piano as listeners to recording. 
%[2] Phenomenological argument: (establishes)
%What will I do with the observation that we do not hear the speaker when we hear the sounds it reproduces. (We can distinguish between a speaker's producing and reproducing a sound, just as we can distinguish between someone saying something and someone rehearsing something.)

% NOTE (I need to note this: In fact I am developing three independently interesting arguments in this paper. I do think they should be developed together. First, there is the argument that we hear a piano when we listen to a recording. Second, there is the argument that this piano must be represented in experience (Local source-representationalism). Third, there is the argument that hearing all non-sound individuals must be represented in experience (Global source-representationalism). Spelling it out in terms of local and global versions helps to tie things together here.)

% SOUND (end)

%PIANO (fold)
\sect I now want to move on and argue that we hear a piano as well. The argument I will develop here is two-step. First I want to develop an argument that has convinced several authors that in ordinary auditory situation we can hear the sources of the sounds we hear. Because we can attend both to the sound and the source, we must suppose that we hear the source as well. I will use this argument to suggest that in listening to a recording, we at least hear more than the sound alone. 
	
% First argument
When during the night you hear a sound coming from your living room, you may immediately realise that the cat has managed to open the kitchen door again. Reluctant to get out of bed at 4am, you first listen more carefully to what is going on, determining whether intervention is imperative. In such a situation, you will not pay much, or even any, heed to the sounds you hear, to their penetrating the chamber door or to their loudness or pitch or timbre. What you attend to is what matters to you at that time: the animal that has invaded your sitting room. You attend to the cat, its whereabouts and current activities \autocite[cf.][p. ?]{nudds2013}. For instance, you may hear the cat scratching the sofa, or, if your unlucky, hear it throwing over a vase (and, inevitably, hear the vase breaking).

Many have been compelled to accept that our ability to attend to more than sounds alone in audition proves that we hear more than sounds alone. What is meant here is not that attending to the cat allows us to hear \emph{that} the cat is in the room. Such `epistemic' hearing of facts may very well be possible, of course. But the point is that the observation, that we can attend both to the sound we hear and to the cat, makes salient that, just as a sound can be an object of audition, a cat can be so too. Hence, \emph{pace} Berkeley, the observation about attention brings home the fact that sound is not the only object of audition.  

If this is right, then we equally have reason to accept that in listening to a recording of a sound we can hear more than the sound alone. This is because, in listening to a recording, we equally are able to attend to more than just the sound. 

Imagine one is an apprentice professional tuner and has been given the job of tuning the cello of a string quartet that will be recorded for a CD release of Schubert's fourteenth string quartet. After the fact, one can surely listen to the recording and attend exclusively to the cello one has tuned. One is able to follow that instrument through the concert, and one's attention may even persist despite the intervals during which the cello has no notes to play.

% Second argument
Someone might suggest there is a disanalogy between hearing one's cat in the living room at night and listening to a recording of a musical instrument. For, in the former case there actually is a cat who directly produces those sounds, while in the latter case what is directly responsible for the sounds we hear is a loudspeaker. Hence, the suggestion would be, the fact that we can attend to something beyond the mere sound could in this case very well mean that we can hear the loudspeaker and not the instrument. 

This objection is not compelling, however. When one of your hifi's midrange speakers, a speaker that reproduces the middle frequencies of the recorded sound, has a defect, this may very well affect your listening experience. Whenever the music approximates a specific frequency range, an annoying tearing or cracking sound can be heard---a sound that is no part of the recording that is played back. Such a cracking sound can be heard, and it can be observed that it does not so much distort the sounds of the piano as interfere with it. It breaks through it, being no longer a reproduced sound from an earlier source, but an original, though unwelcome product of the loudspeaker itself. 

This is important to notice. When we listen to a recording of a piece of music on a speaker set with a broken loudspeaker, we are only then put in a position to attend to the speaker. More precisely, we are only then in a position to attend to the sounds we hear, the instruments that are recorded, and the midway speaker in the centre of the speaker cabinet in the corner of our room. Someone compelled to accept that we can hear a cat because we can attend to it over and above attending to a sound sound be equally compelled to accept that we can hear a piano if we can attend to it over and above attending to a sound and a broken loudspeaker.

%PIANO (end)

% REPRESENTATION (fold)
% Argue: If we hear the piano when we listen to the recording, then we perceive a piano that is not present to us. If we perceive something not present to us, the thing is represented in experience. If we hear the piano when we listen to the recording, then the piano is represented in experience when we listen to the recording. We hear the piano when we listen to the recording. The piano is represented in experience when we listen to the recording.
\sect Perceptual representation may be understood by defining a paradigmatic sufficient condition. If a perceiver perceives some object that is not present to the senses, then the object is represented in perception. Hence, we have a grasp on the phenomenon of perceptual representation, because we are at least able to identify some paradigmatic situations in which the phenomenon occurs. For present purposes, it is enough to think of such occurrences as being facilitated by a capacity we have for representing objects or properties in experience. 

Given this understanding of perceptual representation, we may say that our hearing a piano when we are listening to the recording counts as an instance of it. A piano is represented in our perception of our surroundings. Hence, we can conclude that at least a weak version of source-representationalism is correct: whenever we hear objects `in' or `through' hearing the sounds played back from a recording those objects are merely represented in experience. 

Some may be interested in enquiring into an explanation of this phenomenon. How is perceptual representation of this kind possible? Pointing to a manifestation of a representational capacity we have may not be entirely satisfactory. Perhaps, however, we should take seriously that the sound we hear plays an essential role in the exercise of that capacity. It may be that it is first and foremost that it is the sound we hear that represents a piano to us. If that is right, then we can further enquire what it is about that sound, as an object that is arguably just present in our experience, that makes it a candidate representation? 

At this point I do not want to move in this direction. Whatever deeper account we give of our capacity to have representational experience, I want to argue that we in fact have reason to think that source-representationalism should be understood as a stronger, global thesis about our perception of the sources of sounds. 

% REPRESENTATION (end)

% GENERALISE (fold)
% Argue: If the piano is represented in experience when we listen to the recording, then the piano is represented in experience when we listen to it as an audience. [[WORK: This is because these experiences are subjectively indistinguishable, and belong to the same perceptual kind. ]]
\sect So far we have seen that it is compelling to think that hearing a playback of a recording we can hear both the sound of the recorded instrument and the instrument itself. We also reached the conclusion that, although the sound may simply be present to the mind, the instrument must be represented if we hear it when we listen to a recording of its sound, given that this instrument is absent from the perceptual situation. However, these conclusions have been established in a discussion about hearing playbacks of recorded sounds. Can we generalise these findings to cases where an instrument is actually played to us? 

One thing we must observe here is that listening to an instrument through a recording, and listen to it being played in real life may be indistinguishable for us. Think of those talented musicians in the bowels of London's Underground about which, at some distance, one may be in doubt whether they are actually playing their guitar or just put on a slick recording and merely pretend to play. It is obvious that reflection on one's auditory experience will never put one in a position with certainty to distinguish between these two possible scenarios. This is because for any actual performance there is a conceivable, indiscriminable counterpart that makes use of a recording. 

This is a general point about audition: for every sound one hears, it is always conceivable that one merely hears a recorded sound. If this is right, does it imply that whether one hears the instrument played representationally does not depend on whether the instrument played is actually present? In other words, if for every situation in which one hears a sound's source that is actually present in one's perceptual situation there is a conceivable counterpart where one merely hears a recording, then does it follow that all hearing of such sources is representational? 

We can resolve this issue by considering what imagining hearing a piano consists in. Recall the possible device that generates auditory hallucinations of hearing a piano as a means of `playing back' recordings. This device would put us in a state we could characterise as being similar to hearing a piano, without its being a hearing of a piano. If such a device were turned on, it would potentially mislead people to think they are genuinely hearing  a piano while in fact they are hallucinating. In imagining hearing a piano, however, we could never be so mislead. It is a fact about the imagination that it does not committed one to believe that the situation one imagines is actual. 

Mike Martin has used an important insight about what it is to imagine having an experience to establish that a visual experience of a piano is constitutively dependent on the presence of a piano. He argues that when we imagine seeing the Pacific Ocean we indeed are non-committal about the actuality of the situation we imagine---we at no point think we are actually seeing an ocean---but we are not neutral about what is contained in the situation we imagine. As he writes,

\begin{quote}
Having the visual experience of water puts one in a position which is not neutral with respect to the actual environment as to whether blue water is present or not: that is how we have to characterise what our visual experience is like. Visualising the water puts you in a position of not being neutral with respect to the imagined situation. In visualising the expanse of water, one is not non-committal whether the imagined situation contains a blue expanse of water.\autocite[p. 414]{martin2002aa}	
\end{quote}

In imagining hearing a piano, we should first note that one equally is not committed to the actuality of the imagines situation, and secondly that one  is equally not entirely non-committal about what is contained in the imagined situation. That is, one will take the situation to be one which contains the sound of a piano. 

However, in imagining hearing a piano one may be neutral about whether the situation one imagines contains a piano or not. An explanation for this is that, as we saw, it is possible to hear a piano in situations that do not themselves contain a piano. For every situation in which one hears a sound's source that is actually present in one's perceptual situation there is a conceivable counterpart where one merely hears a recording that is indiscriminable for one. 

We can give another rationale for the lack of dependence on the presence of the piano. Martin explains that ``When one visualises an ocean like the Pacific, one imagines a blue expanse. Reflecting on what one’s act of visualising is like, one can attend only to the blue expanse that one visualises and nothing else. No surrogate or medium for the water or for the blue are evident to one in so imagining.''(413) But as we saw, in hearing a piano one can always attend to the sound of the piano as well. This suggests that the sound may function as a `surrogate' of `mediator' for the piano. It is not obvious that such an aural mediator is always evident to one, but as we saw it is always a possible object for attention.

What we can conclude from this is that the experience of hearing a piano, though constitutively dependent on the presence of the sound of a piano, is not constitutively dependent on the presence of a piano. (An aside. As the ancients saw well, the phenomenology of hearing voices arguably is different. I am open to the suggestion that in imagining hearing Tom's voice we are committed to the imagined situation's containing Tom. Hearing Tom's voice would then be constitutively dependent Tom's presence to one. This would explain why many have characterised the hearing of a voice as a meeting of minds. Source-representationalism is a thesis about the sources of mere sounds, and not about the `sources of voices'.)

% GENERALISE (end)

% EVALUATE (fold)
% The experience of tinnitus does not similarly move us to accept that hearing sounds is representational. But we have seen that we should accept source representationalism: that hearing sources of sounds is representational. 
\sect

For perception to be representational, we earlier noted that it is sufficient if we have an object of perception that is absent from the perceptual scene. Now we see that this is not a necessary condition. Often the objects we hear are present to us, for instance because we see them (assuming naive realism). What is necessary for an aspect of experience to be representational is that the object could be absent. This entails that some object can both be presented and represented in experience. This is interesting to develop.

Emphasise that this argument has no force in the debate between naive realists and representationalists about experience. I find it plausible to think that sounds are not represented in experience, but someone might wish to try to defend the thesis that they are, because perceptual experience as such is representational. If such an argument would succeed, then source representationalism is still true---though I wonder whether the representationalist will use `representation' in exactly the sense we have seen sources are represented in experience. 

For further research, It is a good question how we should approach the hallucinatory case. If we have a hallucination of hearing a piano, should we still say that a piano is represented in experience, or should we say that we have a hallucination of a piano's being represented in experience? 

% EVALUATE (end)

% CONCLUSION (fold)
Here is the main argument again. `(C1)' means that these are conclusions of further arguments I need to offer.

\begin{itemize}
	\item We hear the sound of the piano when we listen to a recording of a piano being played (C1)
	\item We hear a piano when we listen to a recording of a piano being played (C2)
	\item If we hear a piano when we hear€ a recording of a piano being played, then a piano is represented in perception (C3)
	\item If piano is represented in perception, then the perception of a piano does not depend on the presence of a piano (C4) 
	\item If a perception of a piano does not depend on the presence of a piano, then a piano is represented in experience (C5)
	\item If we hear an actual piano being played we have an experience of fundamentally the same kind as listening to a recording of a piano being played (C6)
	\item If we hear an actual piano being played, then the perception of a piano does not depend on the presence of a piano (C7: From C4 and C6)
	\item If we hear an actual piano being played, then a piano is represented in experience (C8: From C5 and C7)
\end{itemize}
% CONCLUSION (end)




%Bibliography: \standardbib just loads a regular bibliography. All files have been loaded in the preamble.
\printbibliography
\end{document}

