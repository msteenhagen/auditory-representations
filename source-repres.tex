% 
%  thoughtsnotthought.tex
%  
%  Created by Maarten Steenhagen on 2013-07-01.
%  Copyright 2013 Maarten Steenhagen. All rights reserved.
% 
\documentclass[sloppy, journal, git, bytitle, dodraft]{humapap}
\usepackage{soul}
\myauthor{Maarten Steenhagen}
\myemail{\\m.steenhagen.09@ucl.ac.uk\\} % Will be used on first page, loaded in author.tex
\mydraft{Draft version}
\theaffiliation{}
\thetitle {source representationalism} % Fist page (no caps is fine)
%\mysubtitle{a second title} % First page
%\mydraft{Draft - do not cite}
\thehtitle{the header} % Title in the header (no caps is fine)
\mydescription{Draft version} % Hovers over the title. Can be left blank (~)
\thejournal{the journal} % Left corner subsequent pages
\theyear{2014}
\thanks{}
\remote{https://bitbucket.org/msteenhagen/humapap/commits/}
\begin{document}
\documenttitle

\begin{abstract}
Source representationalism is the thesis that the sources of the sounds humans hear are, whenever they are heard, represented in experience. I defend this thesis on the basis of considerations about listening to recordings of everyday sounds. 
\end{abstract}	

% INTRODUCTION (fold)
\dropcap{S}{ource representationalism} is a view about auditory perception, according to which the sources of the sounds humans hear are, whenever they are heard, represented in their perceptual experience. When we hear a piano the source representationalist may allow that our auditory experience consists in a perceptual relation to the sound of a piano, yet they will deny that it consists in a perceptual relation to the piano itself. In this paper I will defend source representationalism on the basis of some considerations about listening to recordings of everyday sounds.
% INTRODUCTION (end)

% SOUND (fold)
\sect Consider a situation in which you decide to listen to a stereo recording of Tatiana Nikolayeva's rendition of \emph{Die Kunst der Fuge}, a famous piano performance of J. S. Bach's baroque masterpiece. In contrast to its original audience which found itself in some massive German concert hall, you are currently seated in London, in a small and quiet room with only some books, a writing desk, and a high fidelity speaker set in your vicinity. You know that both the quality of the recording and the calibration of the loudspeakers in the room are superb, and sit back at ease, eyes closed, well aware that you're in for a treat. Carefully you press the button that will start playback. Now what do you hear? 

% You hear the sound of a piano
\sect First of all, you hear the sound of a piano. This particular sound is perceived as soon as it fills the room when the recording starts to play. The sound of the piano may on close attention even sunder into distinct sounds of the keys pressed by Nikolayeva skilled hands. Some of these sounds will be louder than others, some may be high-pitched while others are  of lower audible frequencies.

% Introduces the hallucinatory case used later
Contrast this with the use of the conceivable device which, instead of playing back sounds, causes us to have auditory hallucinations. Playing back a recording on such a device would result your having a series of auditory sensations, sensations that in some respects resemble tinnitus and in other more subjectively salient respects resemble hearing a performance of Bach's \emph{Kunst der Fuge}. Enjoying the recording of Bach's music by means of this device would not involve hearing any sound. 

It should be clear that listening to an ordinary playback of a sound recording of Bach's work does not depend on hallucination. Hence, we may accept that in listening to the recording of a piano being played by Nikolayeva our auditory experience can be one of hearing a sound, and more precisely the sound of a piano.

Someone might accept that we hear sound, but ask what could be meant by saying that now we hear the sound of a piano. Can we really hear the sound of a piano in the absence of any such instrument? I think we can. It has been argued that it is of the nature of sounds that they are reproducible. If this is right, then your speakers would re-produce the particular sounds that could be heard by the audience on the occasion of the actual performance as well. In other words, in hearing the tones reproduced by your loudspeakers, you may be hearing the same sounds that were produced by the piano Nikolayeva played on the evening of the recorded concert. 

For whatever reason, however, some may wish to resist considering sounds to be reproducible, to be abstract in this way. Be that as it may. Even they can agree that in listening to the playback one hears some particulars sounds that, given a well-attuned stereo set, are ideally qualitatively identical to and hence indiscriminable from the sound an actual piano would produce. Hence, I would like to invite those reluctant to take seriously the reproducibility of sounds to keep in mind that wherever the argument that is to follow relies on our hearing the sound of a piano, it is at any time possible to understand this as committing to no more than the claim that we hear a sound that is, at least ideally, indistinguishable from the sound the actually recorded piano produced. 

% SOUND (end)

%PIANO (fold)
\sect From the observation that we hear the sound of a piano, I want to move on and argue that we hear a piano as well. Not only do we hear the sound, we hear its source, the specific instrument that produced it, as well. 

% TODO Rethink the dialectic of the argument for hearing sources. I need to exploit the 'burden' issue. But I need to use the attending issue as additional support of similarity. I need this because I need to make a plasuble case for the seems 'essentially the same' move, and resulting challenge.
The argument I will develop here is two-step. First, I want to make use of an argument that has convinced at least a number of authors that in ordinary auditory situations we can hear the sources of the sounds we hear. The argument is that, because we can attend both to the sound and to the source, we may say that we hear the source as well. I will use this argument to suggest that, also in listening to a recording, we hear a sound's source, or at least hear more than the sound alone. 
	
% First argument
When, during the night, you hear a sound coming from your living room, you may immediately realise that the cat has managed to open the kitchen door again. Reluctant to get out of bed at 4am, you first listen more carefully to what is going on. You do this to determine whether intervention is imperative, hoping of course that you can safely go back to sleep. In such a situation, you will not pay much, or even any, heed to the sounds you hear. You don't attend to them or their penetrating the chamber door or to their loudness or pitch or timbre. What you attend to is what matters to you at that time: the feline occupant of your sitting room. You attend to the cat, to its whereabouts and to its current activities \autocite{nudds2013aa}. For instance, you may hear the cat scratching the sofa, or, if your unlucky, hear it throwing over a vase (and, inevitably, hear the vase breaking).

% TODO Add points about vocabulary and description. Think of Reid and Strawson
Given that we would already be very happy to describe our experience as one of hearing a cat \autocite{urmson1968aa}, the fact that we in addition can distinguish between these two acts of attention removes any doubts that we hear more than sounds alone. \emph{Pace} Berkeley, the observation brings home that sound is not the only object of audition.  

Notice, what is meant here is not that attending to the cat allows us to hear \emph{that} the cat is in the room. Such `epistemic' hearing of facts about your pet may very well be possible of course, and even Berkeley could accept such a claim. The point made, however, is more subtle. It is that attending in hearing both to the sound and to the cat only seems possible if both sound and cat can be objects of audition. If this is right, then we have reason to accept that, similarly, in listening to a recording of a piano concert we can hear more than sound alone. This is because, in listening to such a recording, we equally are able to attend to a piano. % TODO At this point, I should also mention some of the common sense observations about descriptions of recordings

Consider an example. Imagine one is an apprentice professional tuner and has been given the job of tuning the cello of a string ensemble that will be recorded for an album release of Schubert's monumental fourteenth string quartet. Understandably, one will be concerned about whether one did adequate job. Anxiously listening to the resulting recording, one cannot help forgetting all about the string composition, and attends instead exclusively to the cello one has tuned---to those five particular strings that one promised to ring in perfect harmony. It should be clear that such a mode of listening is possible; one is able to attend to and track the instrument through the concert, perhaps even being concerned specifically with no more than a single string. Moreover, one's attention to it may even persist during the brief intervals in which the instrument has no notes to play. If this is right, then it we may equally say that in listening to a recording we hear more than just the sound that is reproduced.

% Second argument
However, someone might suggest that there is a disanalogy between hearing one's cat in the living room at night and listening to a recording of a musical instrument. For, in the former case there actually is a feline presence directly producing those sounds in one's living room, while in the latter case what is directly responsible for the sounds we hear is a mere loudspeaker. Hence, the suggestion would be, the fact that we can attend to something beyond the mere sound could in this case very well indicate no more than that we can hear the loudspeaker and not the instrument. 

This objection is not compelling, however. When one of your hifi's midrange speakers---a speaker that reproduces the middle frequencies of the recorded sound---has a defect, this may very well affect your listening experience. Whenever the music approximates a specific frequency range, an annoying tearing or cracking sound could make itself heard. Such a sound would not be part of the recording that is played back. Such a cracking sound could be heard, and it may be observed that it would not so much distort the sounds of the piano as \emph{interfere} with it; it breaks through the recorded sound, and one no longer merely hears the reproduced sound from an earlier source, but suddenly becomes in addition aware of the original, though unwelcome products of the loudspeaker itself. 

This is important to keep in mind. If we want to understand what it would be to attend in hearing to a loudspeaker, we must consider cases where we listen to a recording on a broken speaker set. In such cases, it should be clear, we may be in a position to attend not only to the sounds we hear and to a piano involved in the musical composition, but also to the midway speaker in the centre of the speaker cabinet in the corner of our room. Someone compelled to accept that we can hear a cat because we can attend to it over and above attending to a sound should be equally compelled to accept that we can hear a piano if we can attend to it over and above attending to a sound and a broken loudspeaker. Hence, our auditory experience when we listen to the recording can be one of hearing a piano.

%PIANO (end)

% REPRESENTATION (fold)
% Argue: If we hear the piano when we listen to the recording, then we perceive a piano that is not present to us. If we perceive something not present to us, the thing is represented in experience. If we hear the piano when we listen to the recording, then the piano is represented in experience when we listen to the recording. We hear the piano when we listen to the recording. The piano is represented in experience when we listen to the recording.
\sect For present purposes perceptual representation may be understood by defining a paradigmatic sufficient condition. If a perceiver perceives some object that is not present to the senses, then the object is represented in perception. In this way we have a grasp on the phenomenon of perceptual representation, because we are at least able to identify some paradigmatic occurrences of the phenomenon. It is enough to think of such occurrences as being facilitated by a capacity of some sort we have for representing objects or properties in experience. 

Now given this understanding of perceptual representation, we may say that, when we are listening to the recording, our hearing a piano counts as an instance of perceptual representation. A piano is represented in our perception of our surroundings, because we hear such an instrument while it is not present to the senses. Hence, we can conclude that at least a weak version of source-representationalism is correct: whenever we hear objects `in' or `through' hearing the sounds played back from a recording, those objects are  represented in experience. 

Some may be interested in enquiring into an explanation of this phenomenon. By what mechanism is perceptual representation of this kind possible? Pointing to a manifestation of a representational capacity we have may not be entirely satisfactory. Perhaps, however, we should take seriously that the sound we hear plays an essential role in the exercise of that capacity. Perceived sounds figuring as cogs in the machinery of representation; it may be that by determining in part the character of our experience it is the sound we hear that represents a piano to us. If that is right, then we can further enquire what it is about that sound, as an object that is arguably just present in our experience, that makes it a candidate representation? 

At this point I do not want to move the enquiry in this direction. Whatever deeper account we give of our capacity to have representational experience, if the previous arguments are along the right lines we do possess such a capacity. I what follows I want to argue that we have reason to think that source-representationalism should be understood as a stronger, global thesis about our perception of the sources of sounds. 

% REPRESENTATION (end)

% GENERALISE (fold)
% Argue: If the piano is represented in experience when we listen to the recording, then the piano is represented in experience when we listen to it as an audience. [[WORK: This is because these experiences are subjectively indistinguishable, and belong to the same perceptual kind. ]]
\sect So far we have seen that it is compelling to think that, in hearing a playback of a recording, we can hear both the sound of the recorded instrument and this instrument itself. Our auditory experience is one both of hearing the sound of a piano and of a piano. We also reached the conclusion that, although the sound may simply be present to the mind, the instrument must be represented if we hear it when we listen to a recording of its sound, given that this instrument is absent from the perceptual situation. However, these conclusions have been established in a discussion about hearing playbacks of recorded sounds. Can we generalise these findings to cases where an instrument is actually played to us? 

One thing we must observe here is that listening to an instrument through a recording, and listen to it being played in real life may be perceptually indistinguishable for us. Think of those talented musicians in the bowels of London's Underground about which, at some distance, one may be in doubt whether they are actually playing their guitar or just have put on a slick recording and are merely pretending to play. Merely reflecting on one's auditory experience can not put one in a position with certainty to distinguish between these two possible scenarios; this is because, for any actual performance, there is a conceivable, indiscriminable counterpart that makes use of a recording. 

This is a general point about audition: for every sound one hears, it is always conceivable that one hears a played back recorded sound. But if that is right, does it imply that whether one hears the instrument played representationally does not depend on whether the instrument played is actually present? In other words, if for every situation in which one hears a sound's source that is actually present in one's perceptual situation there is a conceivable counterpart where one merely hears a recording, then does it follow that all hearing of such sources is representational? 

% TODO Outline the 'sufficient condition' argument

When we sit in our room and do not play a recording, we cannot hear a piano. If we put on the recording, which results in hearing the sound of a piano, we are able to hear a piano. This suggests that hearing the sound of a piano is a sufficient condition on being able to hear a piano. 

But it also suggests that hearing the sound of a piano is a sufficient condition for hearing a piano representationally. For, that is what we are able to do when we play the recording in our room: we are able to hear a piano that is not present to the senses. 

If that is right, then we can observe that this sufficient condition is met also when we are seated in the concert hall during a concert. Also from that situation we hear the sound of a piano, at least when the piano on stage is played. Hence, we can say, hearing that sound is sufficient for hearing a piano, and also that hearing that sound is sufficient for hearing a piano representationally. 

``Hearing the sound of a piano''. That has been a central phrase. We can give three glosses, at least. The first would be a causal one (the source of the auditory information). The second would be a `kind' one: it is a piano-sound: a sound proprietary to pianos. The third would be an intentional one. We are trying to capture the third one. Hearing the sound of a piano means hearing a sound and hearing a piano in it. We could think that this means that one hears a sound, and hears it as the sound of a piano. But this is perhaps unclear. 

The opponent of source-representationalism might suggest that the presence of the piano in the concert situation should not be ignored. They may suggest that this presence makes a genuine difference to our experience. Yet, it is not clear in what this difference could consist. We observed that both experiences are clearly perceptual, as we are unmistakably hearing a sound. Further, we are hearing the same sound. Further, we are able in both cases to attend in hearing to both the sound and the piano. And we are in both cases happy to think of our experience, and describe it, as one of hearing a piano. 

It is not the case that all we know is that the recording case and the concert case are subjectively indistinguishable. They are, by hypothesis. But it is controversial whether these subjective reports are a reliable indicator of sameness of experience. And, in contrast to the argument from illusion, we are not advancing a controversial thesis if we say that in both cases we are aware of the same sensible intermediary. For, it was obvious from the start that such an object of perception was in play: we hear the same sound in both cases. Hence, the argument I am advancing runs neither parallel to the argument from hallucination (because it is obvious both are perceptual), nor parallel to the argument from hallucination (because it is obvious both cases involve awareness of an intermediary). 

% TODO Write new section developing the 'sufficient condition' argument.

% It think the previous conclusion does follow. We can resolve the issue by focusing on the imagination. Consider what imagining hearing a piano consists in. Recall the possible device that generates auditory hallucinations of hearing a piano as a means of `playing back' recordings. This device would put us in a state we could characterise as one similar to hearing a piano, without being a hearing of a piano. If such a device were turned on, however, it is able to mislead some people into think they are genuinely hearing a piano while in fact they are hallucinating. That is the potential grip of hallucination.
% 
% In imagining hearing a piano, however, we could never be so mislead---imagining does not have the same grip on us. If, in your study, you visualise an orchestra, and so imagine seeing an orchestra, you will not believe that there is an orchestra in your study. That must be plain. It is a fact about the imagination that it does not leads one to believe that the situation one imagines is actual. 
% 
% This, however, does not take away that visualising a specific situation does lead one to have specific beliefs about the situation one visualises. Mike Martin takes this to be an important insight about what it is to imagine having an experience, and I think he is right to do so. He argues that, when we imagine seeing the Pacific Ocean, we may remain neutral about the actuality of the situation we imagine but we are not neutral about what is contained in the imagined situation. ``In visualising the expanse of water,'' he writes, ``one is not non-committal whether the imagined situation contains a blue expanse of water'' \autocite[p. 414]{martin2002aa}. 
% 
% Now if this is right, then how do Martin's observations about the visual imagination translate to cases in which we imagine hearing something? The first thing to note is that also In imagining hearing, say, a piano, we are not lead to believe in the actuality of the imagined situation. 
% 
% A second thing we may observe is that imagining hearing a piano one is equally not entirely neutral about what is contained in the imagined situation. Just as visualising the Pacific Ocean commits one to thinking that the scene one imagines contains a blue expanse of water, imagining hearing a piano commits one to thinking that the situation one imagines contains the sound of a piano.
% 
% Yet that is where one's commitments end. Significantly, what one is not committed to when one imagines hearing a piano is thinking that the situation one imagines contains a piano. That this is so can be easily demonstrated. If imagining hearing a piano were to come with the commitment that the situation one imagines contains a piano, then hearing a piano in the absence of a piano would not be possible. Yet, as we saw, it is possible to hear a piano in the absence of such an instrument. Hence, imagining hearing a piano does not commit one to thinking that the situation one imagines contains a piano.
% 
% Martin explains that ``When one visualises an ocean like the Pacific, one imagines a blue expanse. Reflecting on what one’s act of visualising is like, one can attend only to the blue expanse that one visualises and nothing else. No surrogate or medium for the water or for the blue are evident to one in so imagining.''(413) But as we saw, in hearing a piano one can always attend to the sound of the piano as well. This at least suggests that we may think of the sound as some entity that functions as a `surrogate' or `mediator' for the piano we hear. 
% 
% What we can conclude from this is that the experience of hearing a piano, though constitutively dependent on the presence of the sound of a piano, is not constitutively dependent on the presence of a piano. 

% TODO Sharpen up the voices discussion; it is fun but needs a better fit
\sect Although it is only indirectly relevant, an observation about hearing voices might be appropriate. As the ancients saw well, the phenomenology of hearing someone's voice is arguably different. I want to take that seriously. Perhaps hearing a voice simply suffices for hearing that person. A recording of a voice does not so much reproduce the voice, but reproduce the sound of the voice. In that case, though hearing the sound of the voice may be sufficient to hear the person, in meeting someone in real life one would not hear that sound. One would hear the voice. If that is right, then from the observation that one can hear Margaret representationally when one hears the sound of her voice when one listens to a recording, one cannot conclude that one can hear Margaret representationally if one hears her in real life, for one does not hear the sound of her voice when one meets her in real life, but instead simply hears her voice. This is a subtle point, but it might suggest that it is not possible to hear someone's voice in someone's absence. 

Many philosophers have characterised the hearing of a voice as a meeting of minds, and the special phenomenology that seems to have surfaced here might go some way motivating their description. The distinction between hearing voices and hearing sounds that is made in the ancient authors may in fact be explained by the account defended in this paper. Be that as it may. Source-representationalism, as I construe it here, is a thesis about the sources of sounds, and not about the `sources of voices', whatever those may be.

% GENERALISE (end)

% EVALUATE Conclude (fold)
% The experience of tinnitus does not similarly move us to accept that hearing sounds is representational. But we have seen that we should accept source representationalism: that hearing sources of sounds is representational. 
% TODO Rephrase conclusion
\sect This allows us to conclude the argument. % Because hearing a piano, or some other object that is not itself a sound is not constitutively dependent on the presence of a piano or such an other object
, hearing a piano involves the perceptual representation of a piano. We have seen how this follows from a series of claims we may make about the recording of sound, and our listening to it on the one hand, and our imagining hearing pianos or other things on the other. Source-representationalism is a view about auditory perception, according to which the sources of the sounds humans hear are always represented in their perceptual experience. We have come to see that may very well be a correct view about auditory perception.

For perception to be representational, we earlier noted that it is sufficient if we have an object of perception that is absent from the perceptual scene. Now we see that this is not a necessary condition. Often the objects we hear are present to us, for instance because we see them. What is necessary for an aspect of experience to be representational is that the object could be absent. The above argument brings out in what sense the piano could be absent from our experience when we hear it in its presence. What this suggests is that perceptual representation allows for an object to be both presented and represented in experience. 

% TODO Expand on the remarks in the final paragraph, and explain in a bit more detail why this does not point to a general representationalism. 
The argument developed here has no force in the debate between naive realists and representationalists about experience. This should be emphasised. 

The naive realist about experience claims that we at least some times stand in perceptual relations with things in the world. The representationalist about experience denies this, and claims that experience never is a genuine relation with the world. I find it plausible to think that at least some times are perceptually related to the sounds in our environment, and I have loosely assumed this throughout the preceding. In arguing that auditory experience of sources of sounds should not be understood as a perceptual relation to those sources, I here wish to remain neutral about whether auditory experience of sources of sounds should be understood as a perceptual relation to some sound. 

In contrast to the naive realist, representationalists about experience will be committed to giving up neutrality about this point, however. They must suppose that also those sounds are represented in experience. Also on their account, source representationalism will be true, though that truth, that sources of sounds are represented in experience, may come out less spectacular by their standards. But also for representationalists the argument for source-representationalism developed in this paper will be of interest, given that it sheds light on a central concept, perceptual representation, that more often than not remains undefined. 


% EVALUATE Conclude (end)


% 
% \begin{itemize}
% 	\item We hear the sound of the piano when we listen to a recording of a piano being played (C1)
% 	\item We hear a piano when we listen to a recording of a piano being played (C2)
% 	\item If we hear a piano when we hear€ a recording of a piano being played, then a piano is represented in perception (C3)
% 	\item If piano is represented in perception, then the perception of a piano does not depend on the presence of a piano (C4) 
% 	\item If a perception of a piano does not depend on the presence of a piano, then a piano is represented in experience (C5)
% 	\item If we hear an actual piano being played we have an experience of fundamentally the same kind as listening to a recording of a piano being played (C6)
% 	\item If we hear an actual piano being played, then the perception of a piano does not depend on the presence of a piano (C7: From C4 and C6)
% 	\item If we hear an actual piano being played, then a piano is represented in experience (C8: From C5 and C7)
% \end{itemize}
% 
% 


%Bibliography: \standardbib just loads a regular bibliography. All files have been loaded in the preamble.
\printbibliography
\end{document}

