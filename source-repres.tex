% 
%  thoughtsnotthought.tex
%  
%  Created by Maarten Steenhagen on 2013-07-01.
%  Copyright 2013 Maarten Steenhagen. All rights reserved.
% 
\documentclass[sloppy, journal, git, bytitle, dodraft]{humapap}
\usepackage{soul}
\myauthor{Maarten Steenhagen}
\myemail{\\m.steenhagen.09@ucl.ac.uk\\} % Will be used on first page, loaded in author.tex
\mydraft{Draft version}
\theaffiliation{}
\thetitle {source representationalism} % Fist page (no caps is fine)
%\mysubtitle{a second title} % First page
%\mydraft{Draft - do not cite}
\thehtitle{the header} % Title in the header (no caps is fine)
\mydescription{Draft version} % Hovers over the title. Can be left blank (~)
\thejournal{the journal} % Left corner subsequent pages
\theyear{2014}
\thanks{}
\remote{https://bitbucket.org/msteenhagen/humapap/commits/}
\begin{document}
\documenttitle

\begin{abstract}
Source representationalists hold that the sources of the sounds humans hear are, whenever they are heard, represented in experience. I defend this thesis on the basis of considerations about listening to recordings and imagining hearing something. 
\end{abstract}
	
	%Don't forget: represented = perceived & not presented. (Instead of presence in absence, it is perception in absence.)
	
	
	

% INTRODUCTION (fold)
% Make sure this paper is maximally concrete!
\dropcap{S}{ource representationalism} is a view about auditory perception, according to which the sources of the sounds humans hear are, whenever they are heard, represented in their perceptual experience. When we hear a piano the source representationalist may allow that our auditory experience consists in a perceptual relation to the sound of a piano, yet they will deny that it consists in a perceptual relation to the piano itself. In this paper I will defend source representationalism on the basis of some considerations about listening to recordings of everyday sounds, and on the basis of an distinctive feature of imagining hearing everyday objects.
% INTRODUCTION (end)

% SOUND (fold)
\sect Consider a situation in which you decide to listen to a stereo recording of Tatiana Nikolayeva's rendition of \emph{Die Kunst der Fuge}, a famous piano performance of J. S. Bach's baroque masterpiece. In contrast to its original audience which found itself in some massive German concert hall, you are currently seated in London, in a small and quiet room with only some books, a writing desk, and a high fidelity speaker set in your vicinity. You know that both the quality of the recording and the calibration of the loudspeakers in the room are superb, and sit back at ease, eyes closed, well aware that you're in for a treat. Carefully you press the button that will start playback. What do you hear? 

% You hear the sound of a piano
First of all, you hear the sound of a piano. This particular sound is perceived as soon as it fills the room when the recording starts to play back. The sound of the piano may on close attention sunder into distinct sounds of the keys pressed by Nikolayeva skilled hands. Some of these sounds will be louder than others, some may be high-pitched while others are  of lower audible frequencies.

% Introduces the hallucinatory case used later
Contrast this with the use of a device that, instead of playing back sounds, causes us to have auditory hallucinations of hearing the sound of a piano. A playback on such a device would result your having a series of auditory sensations, sensations that in some respects resemble tinnitus and in other more subjectively salient respects resemble hearing a performance of Bach's \emph{Kunst der Fuge}. Enjoying the recording of Bach's music by means of sucha device would not involve hearing any sound. 

It should be clear that listening to an ordinary play-back of a sound recording of Bach's work does not depend on hallucinating any sound. Hence, we should accept that in listening to the recording of a piano being played by Nikolayeva we hear sound, and moreover the sound of a piano.

Someone might ask what could be meant by saying that we hear the sound of a piano when we listen to a mere recording. Can we really hear the sound of a piano in the absence of any piano? I think we can. It has been argued that it is of the nature of sounds that they are reproducible. If this is right, then your speakers would re-produce the particular sounds that could be heard by the audience on the occasion of the actual performance as well. In other words, in hearing the tones reproduced by your loudspeakers, you may be hearing the same sounds that were produced by the piano Nikolayeva played on the evening of the concert. 

For whatever reason, however, some may wish to resist considering sounds to be reproducible, to be abstract in this way. Nonetheless, even they can agree that in listening to the playback one hears some particulars sounds that, given a well-attuned stereo set, are ideally qualitatively identical to and hence indiscriminable from the sound an actual piano would produce.

% SOUND (end)

%PIANO (fold)
\sect I now want to move on and argue that we hear a piano as well. The argument I will develop here is two-step. First I want to develop an argument that has convinced several authors that in ordinary auditory situation we can hear the sources of the sounds we hear. Because we can attend both to the sound and the source, we must suppose that we hear the source as well. I will use this argument to suggest that in listening to a recording, we at least hear more than the sound alone. 
	
% First argument
When during the night you hear a sound coming from your living room, you may immediately realise that the cat has managed to open the kitchen door again. Reluctant to get out of bed at 4am, you first listen more carefully to what is going on, determining whether intervention is imperative. In such a situation, you will not pay much, or even any, heed to the sounds you hear, to their penetrating the chamber door or to their loudness or pitch or timbre. What you attend to is what matters to you at that time: the animal that has invaded your sitting room. You attend to the cat, its whereabouts and current activities \autocite[cf.][p. ?]{nudds2013}. For instance, you may hear the cat scratching the sofa, or, if your unlucky, hear it throwing over a vase (and, inevitably, hear the vase breaking).

Many have been compelled to accept that our ability to attend to more than sounds alone in audition proves that we hear more than sounds alone. What is meant here is not that attending to the cat allows us to hear \emph{that} the cat is in the room. Such `epistemic' hearing of facts may very well be possible, of course. But the point is that the observation, that we can attend both to the sound we hear and to the cat, makes salient that, just as a sound can be an object of audition, a cat can be so too. Hence, \emph{pace} Berkeley, the observation about attention brings home the fact that sound is not the only object of audition.  

If this is right, then we equally have reason to accept that in listening to a recording of a sound we can hear more than the sound alone. This is because, in listening to a recording, we equally are able to attend to more than just the sound. 

Imagine one is an apprentice professional tuner and has been given the job of tuning the cello of a string quartet that will be recorded for a CD release of Schubert's fourteenth string quartet. After the fact, one can surely listen to the recording and attend exclusively to the cello one has tuned. One is able to follow that instrument through the concert, and one's attention may even persist despite the intervals during which the cello has no notes to play.

% Second argument
Someone might suggest there is a disanalogy between hearing one's cat in the living room at night and listening to a recording of a musical instrument. For, in the former case there actually is a cat who directly produces those sounds, while in the latter case what is directly responsible for the sounds we hear is a loudspeaker. Hence, the suggestion would be, the fact that we can attend to something beyond the mere sound could in this case very well mean that we can hear the loudspeaker and not the instrument. 

This objection is not compelling, however. When one of your hifi's midrange speakers, a speaker that reproduces the middle frequencies of the recorded sound, has a defect, this may very well affect your listening experience. Whenever the music approximates a specific frequency range, an annoying tearing or cracking sound can be heard---a sound that is no part of the recording that is played back. Such a cracking sound can be heard, and it can be observed that it does not so much distort the sounds of the piano as interfere with it. It breaks through it, being no longer a reproduced sound from an earlier source, but an original, though unwelcome product of the loudspeaker itself. 

This is important to notice. When we listen to a recording of a piece of music on a speaker set with a broken loudspeaker, we are only then put in a position to attend to the speaker. More precisely, we are only then in a position to attend to the sounds we hear, the instruments that are recorded, and the midway speaker in the centre of the speaker cabinet in the corner of our room. Someone compelled to accept that we can hear a cat because we can attend to it over and above attending to a sound sound be equally compelled to accept that we can hear a piano if we can attend to it over and above attending to a sound and a broken loudspeaker.

%PIANO (end)

% REPRESENTATION (fold)
% Argue: If we hear the piano when we listen to the recording, then we perceive a piano that is not present to us. If we perceive something not present to us, the thing is represented in experience. If we hear the piano when we listen to the recording, then the piano is represented in experience when we listen to the recording. We hear the piano when we listen to the recording. The piano is represented in experience when we listen to the recording.
\sect Perceptual representation may be understood by defining a paradigmatic sufficient condition. If a perceiver perceives some object that is not present to the senses, then the object is represented in perception. Hence, we have a grasp on the phenomenon of perceptual representation, because we are at least able to identify some paradigmatic situations in which the phenomenon occurs. For present purposes, it is enough to think of such occurrences as being facilitated by a capacity we have for representing objects or properties in experience. 

Given this understanding of perceptual representation, we may say that our hearing a piano when we are listening to the recording counts as an instance of it. A piano is represented in our perception of our surroundings. Hence, we can conclude that at least a weak version of source-representationalism is correct: whenever we hear objects `in' or `through' hearing the sounds played back from a recording those objects are merely represented in experience. 

Some may be interested in enquiring into an explanation of this phenomenon. How is perceptual representation of this kind possible? Pointing to a manifestation of a representational capacity we have may not be entirely satisfactory. Perhaps, however, we should take seriously that the sound we hear plays an essential role in the exercise of that capacity. It may be that it is first and foremost that it is the sound we hear that represents a piano to us. If that is right, then we can further enquire what it is about that sound, as an object that is arguably just present in our experience, that makes it a candidate representation? 

At this point I do not want to move in this direction. Whatever deeper account we give of our capacity to have representational experience, I want to argue that we in fact have reason to think that source-representationalism should be understood as a stronger, global thesis about our perception of the sources of sounds. 

% REPRESENTATION (end)

% GENERALISE (fold)
% Argue: If the piano is represented in experience when we listen to the recording, then the piano is represented in experience when we listen to it as an audience. [[WORK: This is because these experiences are subjectively indistinguishable, and belong to the same perceptual kind. ]]
\sect So far we have seen that it is compelling to think that hearing a playback of a recording we can hear both the sound of the recorded instrument and the instrument itself. We also reached the conclusion that, although the sound may simply be present to the mind, the instrument must be represented if we hear it when we listen to a recording of its sound, given that this instrument is absent from the perceptual situation. However, these conclusions have been established in a discussion about hearing playbacks of recorded sounds. Can we generalise these findings to cases where an instrument is actually played to us? 

One thing we must observe here is that listening to an instrument through a recording, and listen to it being played in real life may be indistinguishable for us. Think of those talented musicians in the bowels of London's Underground about which, at some distance, one may be in doubt whether they are actually playing their guitar or just put on a slick recording and merely pretend to play. It is obvious that reflection on one's auditory experience will never put one in a position with certainty to distinguish between these two possible scenarios. This is because for any actual performance there is a conceivable, indiscriminable counterpart that makes use of a recording. 

This is a general point about audition: for every sound one hears, it is always conceivable that one hears a played back recorded sound. If this is right, does it imply that whether one hears the instrument played representationally does not depend on whether the instrument played is actually present? In other words, if for every situation in which one hears a sound's source that is actually present in one's perceptual situation there is a conceivable counterpart where one merely hears a recording, then does it follow that all hearing of such sources is representational? 

It think this does follow. We can resolve the issue by focusing on the imagination. Consider what imagining hearing a piano consists in. Recall the possible device that generates auditory hallucinations of hearing a piano as a means of `playing back' recordings. This device would put us in a state we could characterise as one similar to hearing a piano, without being a hearing of a piano. If such a device were turned on, it is able to mislead some people into think they are genuinely hearing a piano while in fact they are hallucinating. 

In imagining hearing a piano, however, we could never be so mislead. If you in your study visualise an orchestra, and so imagine seeing the orchestra, you will not believe that there is an orchestra in your study. It is a fact about the imagination that it does not leads one to believe that the situation one imagines is actual. 

This however does not take away that visualising a specific situation does lead one to have specific beliefs about the situation one visualises. Mike Martin takes this to be an important insight about what it is to imagine having an experience, and I think he is right to do so. He argues that, when we imagine seeing the Pacific Ocean, we may remain neutral about the actuality of the situation we imagine but we are not neutral about what is contained in the imagined situation. ``In visualising the expanse of water,'' he writes, ``one is not non-committal whether the imagined situation contains a blue expanse of water.''\autocite[p. 414]{martin2002aa}. 

Now if this is right, then how do Martin's observations about the visual imagination translate to cases in which we imagine hearing something? The first thing to note is that also In imagining hearing, say, a piano, we are not lead to believe in the actuality of the imagined situation. 

A second thing we may observe is that imagining hearing a piano one is equally not entirely neutral about what is contained in the imagined situation. Just as visualising the Pacific Ocean commits one to thinking that the scene one imagines contains a blue expanse of water, imagining hearing a piano commits one to thinking that the situation one imagines contains the sound of a piano.

Yet that is where one's commitments end. Significantly, what one is not committed to when one imagines hearing a piano is thinking that the situation one imagines contains a piano. That this is so can be easily demonstrated. If imagining hearing a piano were to come with the commitment that the situation one imagines contains a piano, then hearing a piano in the absence of a piano would not be possible. Yet, as we saw, it is possible to hear a piano in the absence of such an instrument. Hence, imagining hearing a piano does not commit one to thinking that the situation one imagines contains a piano.

Martin explains that ``When one visualises an ocean like the Pacific, one imagines a blue expanse. Reflecting on what one’s act of visualising is like, one can attend only to the blue expanse that one visualises and nothing else. No surrogate or medium for the water or for the blue are evident to one in so imagining.''(413) But as we saw, in hearing a piano one can always attend to the sound of the piano as well. This at least suggests that we may think of the sound as some entity that functions as a `surrogate' or `mediator' for the piano we hear. 

What we can conclude from this is that the experience of hearing a piano, though constitutively dependent on the presence of the sound of a piano, is not constitutively dependent on the presence of a piano. 

\sect As an aside, I feel the need to make an observation about hearing voices. As the ancients saw well, the phenomenology of hearing someone's voice is arguably different. I want to take that seriously, and hence am open to the suggestion that in imagining hearing Margaret's voice we are committed to the imagined situation's containing Margaret. Perhaps it is not possible to hear her voice in her absence. We may of course record a voice, but it is an open question whether we should count a playback of a recorded voice as a voice in its own right. Perhaps we should not. (We do speak of `hearing the sound of someone's voice' as soon as a person is not in our vicinity---perhaps a rare occasion where we count the one pure audibile as a source of another.) 

Many philosophers have characterised the hearing of a voice as a meeting of minds, and the special phenomenology that seems to have surfaced here might go some way motivating their description. Be that as it may. Source-representationalism, as I construe it here, is a thesis about the sources of sounds, and not about the `sources of voices', whatever those may be.

% GENERALISE (end)

% EVALUATE Conclude (fold)
% The experience of tinnitus does not similarly move us to accept that hearing sounds is representational. But we have seen that we should accept source representationalism: that hearing sources of sounds is representational. 
\sect This allows us to conclude the argument. Because hearing a piano, or some other object that is not itself a sound is not constitutively dependent on the presence of a piano or such an other object, hearing a piano involves the perceptual representation of a piano. We have seen how this follows from a series of claims we may make about the recording of sound, and our listening to it on the one hand, and our imagining hearing pianos or other things on the other. Source-representationalism is a view about auditory perception, according to which the sources of the sounds humans hear are always represented in their perceptual experience. We have come to see that may very well be a correct view about auditory perception.

For perception to be representational, we earlier noted that it is sufficient if we have an object of perception that is absent from the perceptual scene. Now we see that this is not a necessary condition. Often the objects we hear are present to us, for instance because we see them. What is necessary for an aspect of experience to be representational is that the object could be absent. The above argument brings out in what sense the piano could be absent from our experience when we hear it in its presence. What this suggests is that perceptual representation allows for an object to be both presented and represented in experience. 

The argument developed here has no force in the debate between naive realists and representationalists about experience. This should be emphasised. I find it plausible to think that sounds are not represented in experience, and I have loosely assumed this throughout the preceding. Yet, some are committed to the thesis that sounds are represented in experience, because they think perceptual experience as such is representational. Also on their account, source representationalism will be true, though its truth may come out less spectacular by their standards. But also for representationalists the argument for source-representationalism developed in this paper will be of interest, given that it sheds light on a central concept, perceptual representation, that more often than not remains undefined. 


% EVALUATE Conclude (end)


% 
% \begin{itemize}
% 	\item We hear the sound of the piano when we listen to a recording of a piano being played (C1)
% 	\item We hear a piano when we listen to a recording of a piano being played (C2)
% 	\item If we hear a piano when we hear€ a recording of a piano being played, then a piano is represented in perception (C3)
% 	\item If piano is represented in perception, then the perception of a piano does not depend on the presence of a piano (C4) 
% 	\item If a perception of a piano does not depend on the presence of a piano, then a piano is represented in experience (C5)
% 	\item If we hear an actual piano being played we have an experience of fundamentally the same kind as listening to a recording of a piano being played (C6)
% 	\item If we hear an actual piano being played, then the perception of a piano does not depend on the presence of a piano (C7: From C4 and C6)
% 	\item If we hear an actual piano being played, then a piano is represented in experience (C8: From C5 and C7)
% \end{itemize}
% 
% 


%Bibliography: \standardbib just loads a regular bibliography. All files have been loaded in the preamble.
\printbibliography
\end{document}

