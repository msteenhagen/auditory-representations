% 
%  thoughtsnotthought.tex
%  
%  Created by Maarten Steenhagen on 2013-07-01.
%  Copyright 2013 Maarten Steenhagen. All rights reserved.
% 
\documentclass[sloppy, journal, git, bytitle, dodraft]{humapap}
\usepackage{soul}
\myauthor{Maarten Steenhagen}
\myemail{\\m.steenhagen.09@ucl.ac.uk\\} % Will be used on first page, loaded in author.tex
\mydraft{Draft version}
\theaffiliation{}
\thetitle {source representationalism} % Fist page (no caps is fine)
%\mysubtitle{a second title} % First page
%\mydraft{Draft - do not cite}
\thehtitle{the header} % Title in the header (no caps is fine)
\mydescription{Draft version} % Hovers over the title. Can be left blank (~)
\thejournal{the journal} % Left corner subsequent pages
\theyear{2014}
\thanks{}
\remote{https://bitbucket.org/msteenhagen/humapap/commits/}
\begin{document}
\documenttitle

\begin{abstract}
Source representationalism is the thesis that the sources of the sounds humans hear, whenever they are heard, are represented in experience. I defend this thesis on the basis of considerations about listening to sound recordings. I suggest that the observation that listening to a recording and to an actually produced sound could be indistinguishable, combined with the observation that in both cases a sufficient condition for being able representationally to experience a sound's source is met, should lead us to accept source representationalism as a global thesis about audition. 
\end{abstract}	

% INTRODUCTION (fold)
\dropcap{S}{ource representationalism} is a view about auditory perception, according to which the sources of the sounds we hear, whenever they are heard, are represented in our perceptual experience. When we hear a piano being played, the source representationalist may accept that our auditory experience consists in a perceptual relation to the sound of a piano, yet they will deny that it consists in a perceptual relation to that piano itself. In this paper I defend source representationalism on the basis of some considerations about listening to audio recordings.
% INTRODUCTION (end)

% SOUND (fold)
\sect Consider the situation in which you listen to a stereo recording of Tatiana Nikolayeva's rendition of Bach's baroque masterpiece \emph{Die Kunst der Fugue}. In contrast to an actual audience, which could have found itself in some massive German concert hall, you are currently seated in London, in a small and quiet room with only a chair, writing desk and a high fidelity speaker set in your vicinity. You know that both the quality of the recording and the calibration of the loudspeakers in the room are superb, and sit back at ease, eyes closed, well aware that you're in for a treat. Carefully you press the button that will start playback. Now what do you hear? 

% You hear the sound of a piano
\sect First of all, you can hear the sound of a piano. This particular sound becomes perceptible to one as soon as it fills the room when the recording starts to play. On closer attention to Nikolayeva performance, the sound may begin to decompose into individual sounds of piano keys pressed by her skilled hands, some of which may be louder or higher in pitch than others. Let us assume all such sounds you can hear are more or less determinate instances of the sound of a piano.

% Introduces the hallucinatory case, used to make clear our case is a genuine perception.
Contrast this experience with one in which one of those conceivable devices is used, which, instead of playing back sounds, cause us to have auditory hallucinations. Playing back a recording on such a device would result your having a series of auditory sensations, sensations that in some respects resemble tinnitus and in other and more subjectively salient respects resemble hearing a performance of Bach's compositions. Enjoying a piano recording means of such a device would not involve hearing the sound of a piano, and instead work through auditory hallucinations of hearing such a sound. 

It should be clear that listening to an ordinary playback of a sound recording of Bach's work does not in any way depend on hallucination. It is a perception of sound; and some may want to say, more specifically, that it consists in a perceptual relation to sounds of a piano that trickle into the room after playback of the recording has started.

Someone might ask, What could be meant by saying that here we hear the sound of a piano? Can we really hear the sound of a piano in the absence of any such instrument? I think we can. Some argue that it is of the nature of sounds that they are reproducible \autocite{martin2012aa}. If this is right, then your speakers could re-produce the particular sounds that were produced by a piano on the occasion of Nikolayeva's playing. 

For whatever reason, however, some may wish to resist considering sounds to be reproducible and abstract in this way. Yet even they can agree that in listening to the playback one hears some particular sounds that, given a well-attuned stereo set, are qualitatively identical to and hence indiscriminable from the sound an actual piano would produce. I would like to invite those reluctant to accept that sounds are reproducible to keep in mind that, wherever the argument that is to follow relies on our hearing the sound of a piano, it is at any time possible to understand this as committing to no more than the claim that we hear a sound that, at least ideally, is qualitatively identical to the sound an actually played piano could have produced. 

% SOUND (end)

%PIANO (fold)
\sect From the observation that we can hear the sound of a piano when we listen to the recording I now want to move on, and argue that we can hear a piano as well. 

In general, that we can hear more than sounds alone is by many authors taken to be undeniable. J. O. Urmson writes, 
\begin{quote}
Clearly one may see, hear, feel, smell, and taste physical objects like motor cars and apples; it requires ingenious stage setting to make `I hear an apple' or `I taste a motor car' sound natural \lips but one may certainly hear a motor car or taste an apple.\autocite[p. 117]{urmson1968aa}
\end{quote} Just as we can see a magpie, or taste the vinegar in our salad, we can hear things that are not sounds: people in the corridor, aeroplanes overhead, a speeding Volkswagen \autocite[cf.][p. 10-11]{heidegger1977aa}. There simply is no reason to side with Berkeley and adopt the restrictive view that in general sounds are all we hear. 

Though it certainly is one, notice that this observation is not merely a report of common sense; it has a compelling phenomenological basis as well. Both sounds and sources can be made object of attention, and on occasions even compete for it. An increasing number of authors emphasises that the possibility to attend to things that are not sounds reveals that the objects of audition include more than sounds alone \autocite[See the papers in][]{ocallaghan2009aa}.  

Imagine, during the night you hear a sound coming from your living room. You immediately realise that it's the cat who managed to open the kitchen door again. Reluctant to get out of bed at 4 a.m., you first listen more carefully to what is going on. In that situation you will not pay much heed to the sounds you hear; of course you could try to set yourself to attend to them, or to their loudness or pitch or timbre, but you don't, because what your attention in hearing will go out to is what matters to you at that time: the feline threat to designer furniture \autocite[cf.][]{nudds2007aa}. You attentively listen to the cat; you may listen to the cat scratching the sofa, or, if your unlucky, hear it throwing over a vase (and, inevitably, hear the vase breaking).

In hearing a sound, we often can attend to, and hence listen to and hear the things the sounds is of. \emph{Pace} Berkeley, not only would we be very happy to describe the experience described above as one of hearing a cat, the fact that we can also distinguish between these two acts of attention, proves that we can hear more than sounds alone.

This means more than that we are able to hear \emph{that} they are the sounds of a cat, or perhaps \emph{that} the cat is in the room. Such `epistemic' hearing of facts about sounds or pets does of course occur. But the point is more subtle: it is that attentively listening to the sound and to the cat only seems possible if both sound and cat can be heard. Just as Urmon's apple, your cat, as the source of the sound you hear, can be an object of audition. 

If this is right, then by the same reasoning we may say that in listening to a recording we can hear a sound's source. Consider an example. Emelia is an expert instrument tuner who just tuned the cello of a string ensemble that will be recorded performing Schubert's monumental fourteenth string quartet. However, afterwards she realises she may have used a damaged tuning fork. Understandably, Emelia now will be concerned about whether the cello was properly tuned. Anxiously listening to the resulting registration, Emelia forgets entirely about the details of Schubert's moving composition and only listens attentively to the cello she tuned. 

Emelia's way of listening to the recording is conceivable, and exemplifies a mode of listening that is common in a world in which sound recordings are ubiquitous. Hence, with regard to the possibility of listening attentively to one's cat or to a cello, whether one hears a recording simply does not matter.

% Second argument: this touches on something Mike writes in his handout
Someone might object that, where in the former example there actually is a cat producing those sounds, what is directly responsible for the sounds of a recording is a mere loudspeaker. Hence, the objection would run, the fact that one can attend to something beyond the mere sound of a recording suggests just that we can hear the loudspeaker that reproduces it. 

This objection is not compelling, however. As every audiophile will attest, loudspeakers are inaudible if they function well. For an grasp on what it would be hear a loudspeaker we must consider cases where we the device has a noticeable defect. Imagine, whenever a playback approximates a specific frequency range, an annoying tearing sound makes itself heard. A speaker is bust, and its sound now interferes with the sound of the music; the cracking stands out in experience as not belonging to the recording itself. This is what it is like to hear a loudspeaker. It should be clear that, instead of setting the standard for listening to recorded sounds, such an experience signals that equipment needs to be replaced. 


%PIANO (end)

% REPRESENTATION (fold)
% Argue: If we hear the piano when we listen to the recording, then we perceive a piano that is not present to us. If we perceive something not present to us, the thing is represented in experience. If we hear the piano when we listen to the recording, then the piano is represented in experience when we listen to the recording. We hear the piano when we listen to the recording. The piano is represented in experience when we listen to the recording.
\sect The following is a paradigmatic sufficient condition for perceptual representation:
\begin{description}
	\item[Perceptual representation] If a perceiver perceives some object that is not present to the senses, then the object is represented in perception.
\end{description}
With this we have a grasp on the phenomenon of perceptual representation, because it allows us to identify some paradigmatic occurrences. In particular, we may now say that, if we are merely playing back a recording of a piano, then our hearing a piano counts as an instance of perceptual representation because we hear the instrument while it is not present to the senses. Hearing the sound of a piano suffices for being able to hear a piano representationally.

% TODO Should I rephrase `hearing' as having perceptual experience of? Or just experience of?   

Assuming that there is nothing particularly peculiar about piano's, we may now accept a weak version of source representationalism; whenever we hear a source when we merely listen to a reproduction of its sound, the source is  represented in experience, because we hear it while it is not present to the senses.

Some may be interested in enquiring into an explanation of this phenomenon. By what mechanism is perceptual representation of this kind possible? Merely mentioning a capacity for representational perception may not be entirely satisfactory. Yet, we can point to more. Clearly, the heard sound plays a determining role in the exercise of this capacity---just as other perceptual capacities, the capacity for perceptual representation is a \emph{reactive} one \autocites[][]{nietzsche1887aa}[see][p. 4]{kalderon2012ab}. A perceived sound figures as an essential cog in the machinery of representation in auditory perception: by determining in part the character of our experience, we may expect it to be the heard sound that represents a piano to us. This would naturally lead us to ask what is it is about that sound, as an object of experience, that lets it take on such a representative attire.

Here I do not want to move the enquiry in this direction. Whatever deeper account we give of the nature of sounds, or of perceptual representations more generally, if the previous arguments are along the right lines, we do possess a psychological capacity to make use of them. In what follows I want to show that source representationalism should be accepted as a stronger, global thesis about auditory perception.

% REPRESENTATION (end)

% GENERALISE (fold)
% Argue: If the piano is represented in experience when we listen to the recording, then the piano is represented in experience when we listen to it as an audience. [[WORK: This is because these experiences are subjectively indistinguishable, and belong to the same perceptual kind. ]]
\sect We saw that we can hear both the sound of a piano and a piano in listening to a recording of Bach's \emph{The Art of Fugue}. We reasoned that in such circumstances, although the sound may simply be present to the mind, the piano we hear must be represented in auditory experience. This was because the circumstances enabled us to hear such an instrument in its absence. Is this representational dimension an aspect of perceptual experience when an instrument is actually played to us? 

% Indistinguishability
First, we may note that listening to a recording may for us be perceptually indistinguishable from listening to an instrument being played in real life, and vice versa. Think of those talented musicians in the bowels of London's Underground about which, at some distance, one may be in doubt whether they are actually playing their guitar or are just playing back a slick recording. In such a state of doubt, reflecting on one's auditory experience may not be sufficient to distinguish between the two possible scenarios, since for any actual performance, there is a conceivable, indiscriminable counterpart that makes use of a recording. 

This is a general point about audition. For every instrument, aeroplane or midnight brawl one hears, it is always conceivable that one has such an experience because one hears sounds reproduced via a recording. What makes hearing an actual fight outside one's bedroom window and a recorded one potentially indistinguishable? To this question we can give a clear positive answer: both consist in an unproblematic perceptual awareness of the sound of a brawl. Hearing a brawl fight via a recording and hearing a brawl because a pub fight takes place just below one's half-opened window can be indistinguishable because in both cases we could be perceptually aware of the exact same sound, a sound that in each case could have the same impact on the character of our experience.

Our earlier discussion suggested that one's hearing the sound of a piano is a sufficient condition for one to be able to hear a source representationally. We may be in the dark about what it is about the sound we hear that makes this possible; but it nonetheless enables representational perception to occur. But if it is a sufficient condition for such an experience, then also when we hear a sound when it is actually produced by a fighting mob or by a concert piano are we in a position to hear its source representationally. 

Hearing a piano via a recording and hearing a piano while listening to a recital are both cases of perception. It may be said that in both cases we are perceptually related to the same thing: the sound of a piano. In both cases, it is the sound of the piano that in part determines the character of our auditory experience. Moreover, we know that just on the basis of having an experience with such a character, it may be impossible to tell just on the basis of what we hear whether we are hearing a piano that is actually present or a piano that is merely recorded. This suggests that the actuality of a piano in our surroundings makes no contribution to the character or nature of auditory phenomenology. 

Although it may be true that a piano is present in a concert venue we are seated in, we have no reason to consider the piano as present in our auditory experience.  If we are seated in a concert hall at a recital and hear a piano through hearing the sound of a piano, we may say that the instrument is none the less represented in our experience.  


% There are two points we should mention. First of all, it is not said that hearing the sound suffices for hearing the piano; it may very well be that in order to hear the piano we must recognise the sound as the sound of a piano. Second, it is not said that every sound of a piano will allow for such recognition. Hence, if recognising the sound is necessary for hearing a piano representationally, we should expect that there only will be a range of cases in which hearers will actually be able to exploit the representational possibilities that perceptual situation offers. However, in the cases we are considering, we may suppose that the sound we hear when we listen to the recording is qualitatively identical to the sound the audience heard when the piece was performed. Hence, we can assume that hearing this sound is a sufficient condition for hearing a piano, at least for those perceivers who have the relevant recognitional capacities if needed. 

\sect  Source representationalism, a specific position in the philosophy of perception, is the view according to which the sources of the sounds we hear are, whenever we hear them, represented in experience. Reflecting on listening to things through recordings, and acknowledging that hearing recordings can be subjectively indistinguishable from other cases of hearing, has opened up a compelling line of defence of the source representationalist claim. Whenever we hear a piano, or some other sound source, the piano, or other source is represented in experience. 

% If this is correct, then the auditory world is a peculiar one. In part it consists of the sounds that may be present in our experience, in part it consists of a mixed bag of represented instruments, neighbours, air conditioners, taxis, and planes flying overhead. It is an interesting question how the represented aspects of our auditory environment relate to, and perhaps overlap with, the concrete material reality vision or touch is commonly taken to make present to us.


% ``Hearing the sound of a piano''. That has been a central phrase. We can give three glosses, at least. The first would be a causal one (the source of the auditory information). The second would be a `kind' one: it is a piano-sound: a sound proprietary to pianos. The third would be an intentional one. We are trying to capture the third one. Hearing the sound of a piano means hearing a sound and hearing a piano in it. We could think that this means that one hears a sound, and hears it as the sound of a piano. But this is perhaps unclear. 


% It is not the case that all we know is that the recording case and the concert case are subjectively indistinguishable. They are, by hypothesis. But it is controversial whether these subjective reports are a reliable indicator of sameness of experience. And, in contrast to the argument from illusion, we are not advancing a controversial thesis if we say that in both cases we are aware of the same sensible intermediary. For, it was obvious from the start that such an object of perception was in play: we hear the same sound in both cases. Hence, the argument I am advancing runs neither parallel to the argument from hallucination (because it is obvious both are perceptual), nor parallel to the argument from hallucination (because it is obvious both cases involve awareness of an intermediary). 

% TODO Write new section developing the 'sufficient condition' argument.

% It think the previous conclusion does follow. We can resolve the issue by focusing on the imagination. Consider what imagining hearing a piano consists in. Recall the possible device that generates auditory hallucinations of hearing a piano as a means of `playing back' recordings. This device would put us in a state we could characterise as one similar to hearing a piano, without being a hearing of a piano. If such a device were turned on, however, it is able to mislead some people into think they are genuinely hearing a piano while in fact they are hallucinating. That is the potential grip of hallucination.
% 
% In imagining hearing a piano, however, we could never be so mislead---imagining does not have the same grip on us. If, in your study, you visualise an orchestra, and so imagine seeing an orchestra, you will not believe that there is an orchestra in your study. That must be plain. It is a fact about the imagination that it does not leads one to believe that the situation one imagines is actual. 
% 
% This, however, does not take away that visualising a specific situation does lead one to have specific beliefs about the situation one visualises. Mike Martin takes this to be an important insight about what it is to imagine having an experience, and I think he is right to do so. He argues that, when we imagine seeing the Pacific Ocean, we may remain neutral about the actuality of the situation we imagine but we are not neutral about what is contained in the imagined situation. ``In visualising the expanse of water,'' he writes, ``one is not non-committal whether the imagined situation contains a blue expanse of water'' \autocite[p. 414]{martin2002aa}. 
% 
% Now if this is right, then how do Martin's observations about the visual imagination translate to cases in which we imagine hearing something? The first thing to note is that also In imagining hearing, say, a piano, we are not lead to believe in the actuality of the imagined situation. 
% 
% A second thing we may observe is that imagining hearing a piano one is equally not entirely neutral about what is contained in the imagined situation. Just as visualising the Pacific Ocean commits one to thinking that the scene one imagines contains a blue expanse of water, imagining hearing a piano commits one to thinking that the situation one imagines contains the sound of a piano.
% 
% Yet that is where one's commitments end. Significantly, what one is not committed to when one imagines hearing a piano is thinking that the situation one imagines contains a piano. That this is so can be easily demonstrated. If imagining hearing a piano were to come with the commitment that the situation one imagines contains a piano, then hearing a piano in the absence of a piano would not be possible. Yet, as we saw, it is possible to hear a piano in the absence of such an instrument. Hence, imagining hearing a piano does not commit one to thinking that the situation one imagines contains a piano.
% 
% Martin explains that ``When one visualises an ocean like the Pacific, one imagines a blue expanse. Reflecting on what one’s act of visualising is like, one can attend only to the blue expanse that one visualises and nothing else. No surrogate or medium for the water or for the blue are evident to one in so imagining.''(413) But as we saw, in hearing a piano one can always attend to the sound of the piano as well. This at least suggests that we may think of the sound as some entity that functions as a `surrogate' or `mediator' for the piano we hear. 
% 
% What we can conclude from this is that the experience of hearing a piano, though constitutively dependent on the presence of the sound of a piano, is not constitutively dependent on the presence of a piano. 

% TODO Sharpen up the voices discussion; it is fun but needs a better fit
% \sect Although it is only indirectly relevant, an observation about hearing voices might be appropriate. Already the ancients emphasised the special nature of voice, and I want to take seriously the possibility that the perception of voice brings in complexities or may even be an exception to the representational status of who's voice it is in audition. For Aristotle, as Polansky brings out, ``voice is not just any striking of air but the subtle striking under the control of soul'' \autocite[p. 300]{polansky2007aa}. 

% Given that voice and sound are both audibilia, there are two contending ways of thinking about voice. The first is that we take voice to be a species of sound. The second is that we take sound and voice both as distinct species of audibilia. Aristotle, at least in \emph{De Anima}, seems to opt for the former line (000a11). What matters for our present enquiry is that both ways allow hearing voice to make for an exception to source representationalism. This is because in both ways it could be suggested that hearing someone's voice suffices for standing in a perceptual relation to that person. Let me explain these in turn.

% 2. If the second option is preferred---and Aristotle in De Sensu seems to come close to expressing such a preference---then it is not clear whether in hearing someone's voice our perception of the person, or of the intentions expressed in their utterance, is representational. It might be suggested that, in contrast to sound, the person or their intentions are an intrinsic part of the voice. This would suggest that a voice is not a product of a person, as a sound is, but belongs to the person in a more intimate sense. If that would be so, it is not unreasonably to expect that hearing someone's voice is a perceptual relation to someone else. This might be the point I need to distill from this discussion. 

%Voice in humans symbolizes their thought (DI 16a3–9; see Polansky and Kuczewski 1990 and cf. Plato Theaet. 206d). 

% Perhaps hearing a voice simply suffices for hearing that person. A recording of a voice does not so much reproduce the voice, but reproduce the sound of the voice. In that case, though hearing the sound of the voice may be sufficient to hear the person, in meeting someone in real life one would not hear that sound. One would hear the voice. If that is right, then from the observation that one can hear Margaret representationally when one hears the sound of her voice when one listens to a recording, one cannot conclude that one can hear Margaret representationally if one hears her in real life, for one does not hear the sound of her voice when one meets her in real life, but instead simply hears her voice. This is a subtle point, but it might suggest that it is not possible to hear someone's voice in someone's absence. 

% Many philosophers have characterised the hearing of a voice as a meeting of minds, and the special phenomenology that seems to have surfaced here might go some way motivating their description. The distinction between hearing voices and hearing sounds that is made in the ancient authors may in fact be explained by the account defended in this paper. Be that as it may. Source-representationalism, as I construe it here, is a thesis about the sources of sounds, and not about the `sources of voices', whatever those may be.

% GENERALISE (end)

% EVALUATE Conclude (fold)
% The experience of tinnitus does not similarly move us to accept that hearing sounds is representational. But we have seen that we should accept source representationalism: that hearing sources of sounds is representational. 
% TODO Rephrase conclusion
% \sect For perception to be representational, we earlier noted that it is sufficient if we have an object of perception that is absent from the perceptual scene. Now we see that this is not a necessary condition. Often the objects we hear are present to us, for instance because we see them. What is necessary for an aspect of experience to be representational is that the object could be absent. The above argument brings out in what sense the piano could be absent from our experience when we hear it in its presence. What this suggests is that perceptual representation allows for an object to be both presented and represented in experience. 

% TODO Expand on the remarks in the final paragraph, and explain in a bit more detail why this does not point to a general representationalism. 
\sect Notice that the argument developed here has no force in the debate between `Naive Realists' and `Representationalists' about perception. This should be emphasised. 

The Representationalist maintains that all perceptual experience is representational. Perceiving \emph{x} just is perceiving \emph{x} representationally; whatever it is we see, hear or feel, it is represented in experience when we see, hear or feel it. Naïve Realists, on the other hand, deny this. The Naïve Realist maintains that not all perceiving is representational.

Source representationalism is a view according to which at least some aspects of sense experience are representational. Source representationalists would disagree with the Representationalist if the former assume in addition that experience of sound can consist in a perceptual relation, something that could only obtain in the actual presence of the sound. In the previous I have loosely made that assumption. However, making this additional assumption is optional. Nowhere does the argument of this paper depend on it. Hence, source representationalism as such remains silent on whether all perceptual experience is representational.

Yet, if the argument of this paper is sound, it does suggests that, either way, fully understanding the scope of perceptual experience should force us to acknowledge the role perceptual representations can fulfil in it.



% EVALUATE Conclude (end)


% 
% \begin{itemize}
% 	\item We hear the sound of the piano when we listen to a recording of a piano being played (C1)
% 	\item We hear a piano when we listen to a recording of a piano being played (C2)
% 	\item If we hear a piano when we hear€ a recording of a piano being played, then a piano is represented in perception (C3)
% 	\item If piano is represented in perception, then the perception of a piano does not depend on the presence of a piano (C4) 
% 	\item If a perception of a piano does not depend on the presence of a piano, then a piano is represented in experience (C5)
% 	\item If we hear an actual piano being played we have an experience of fundamentally the same kind as listening to a recording of a piano being played (C6)
% 	\item If we hear an actual piano being played, then the perception of a piano does not depend on the presence of a piano (C7: From C4 and C6)
% 	\item If we hear an actual piano being played, then a piano is represented in experience (C8: From C5 and C7)
% \end{itemize}
% 
% 


%Bibliography: \standardbib just loads a regular bibliography. All files have been loaded in the preamble.
\printbibliography
\end{document}

