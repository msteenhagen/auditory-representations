% 
%  thoughtsnotthought.tex
%  
%  Created by Maarten Steenhagen on 2013-07-01.
%  Copyright 2013 Maarten Steenhagen. All rights reserved.
% 
\documentclass[sloppy, journal, git, bytitle, dodraft]{humapap}
\usepackage{soul}
\myauthor{Maarten Steenhagen}
\myemail{\\m.steenhagen.09@ucl.ac.uk\\} % Will be used on first page, loaded in author.tex
\mydraft{Draft version}
\theaffiliation{}
\thetitle {source representationalism} % Fist page (no caps is fine)
%\mysubtitle{a second title} % First page
%\mydraft{Draft - do not cite}
\thehtitle{the header} % Title in the header (no caps is fine)
\mydescription{Draft version} % Hovers over the title. Can be left blank (~)
\thejournal{the journal} % Left corner subsequent pages
\theyear{2014}
\thanks{}
\remote{https://bitbucket.org/msteenhagen/humapap/commits/}
\begin{document}
\documenttitle

% \begin{abstract}That some objects are, as the naive realist has it, presented in perception does not rule out that some objects of perception are represented. I show that at least some plausible versions of naive realism to accept that audition essentially tends towards representational perception. The argument for this is that,  on the one hand, our capacity for auditory perception is for the sake of hearing the sources of sounds, while on the other those sources, if they are not sounds, cannot be constituents of auditory perception. This implies that audition essentially has a tendency to afford perception of objects that need not be constituents of experience, which in turns suggests that such perceptions defy the non-representationalist analysis offered by the naive realist and, hence, must be represented in experience.\end{abstract}
	
	%Don't forget: represented = perceived & not presented. (Instead of presence in absence, it is perception in absence.)
	
	
	

% INTRODUCTION (fold)
% Make sure this paper is maximally concrete!
\dropcap{S}{ource representationalism} is a view about auditory perception, according to which the sources of the sounds humans hear are always represented in their perceptual experience. When we listen to and hear a piano, our auditory experience may consist in a perceptual relation to its sound, it can not consist in a perceptual relation to the piano itself. In this paper I will defend source representationalism on the basis of some considerations about listening to recordings of everyday sounds. 
% INTRODUCTION (end)

% SOUND (fold)
% Introduce key example. Argue: We hear the sound of the piano when we listen to the recording. Either the same sound or a similar sound - this depends on one's view of recording and playback of sound. 
\sect Consider a situation in which you decide to listen to a stereo recording of Tatiana Nikolayeva's rendition of \emph{Die Kunst der Fuge}, a famous piano performance of J. S. Bach's baroque masterpiece that took place on a winter evening in 1953. In contrast to its original audience which found itself in the massive Building in City, you are currently seated in London, in a small and quiet room with only some books, a writing desk, and a high fidelity speaker set in your vicinity. You know that both the quality of the recording and the calibration of the loudspeakers in the room are superb, and sit back at ease, eyes closed, well aware that you're in for a treat. Carefully you press the button that will start playback. What do you hear? 

% You hear the sound of a piano
First of all, you hear the sounds of a piano. These particular sounds are perceived as soon as they fill the room when the recording starts to play back. Some will be loud, while others are muted; some may be high-pitched while others of lower audible frequencies. But is is undeniable that you can hear these sounds.   

% Introduces the hallucinatory case used later
Contrast this with the use of a device that, instead of playing back sounds, causes us to have auditory hallucinations of the sounds of a piano. A playback on such a device would result your having a series of auditory sensations, sensations a bit like tinnitus but a lot more complex, and that resemble hearing a performance of, say, Bach's \emph{Kunst der Fuge}. `Listening' to Bach's music here would not involve hearing any sound. Yet, it should be clear that ordinary experience of playing back of sound recordings does not depend on hallucinating sound. 

% Spells out qualms about reproducability: takes a neutral stance. (But keep an eye on this!) The point I need to hold on to is that we can hear the sound of a piano, and so the resemblance view must be developed as giving a gloss of that phrase as well. (Perhaps: what matters is that we hear a particular, whether abstract or concrete. The particular determines character. This particular could occur both in recording and actual cases.)
%We can use the idea of a prorietary sound here: a badly oiled revolving door may make the sound of a dying animal; sounds like a dying animal. The speaker sounds like a piano: this would not be a correct description. We do not hear the speaker.
If we say that, seated in our room, we hear the sound of a piano, what does this mean? Some have suggested that it is of the nature of sounds that they are reproducible. If they are right, then your speakers would \emph{re}produce the particular sounds that could be heard by the audience on the occasion of the actual performance as well. In other words, in hearing the tones reproduced by your loudspeakers, you may be hearing the same sounds that were produced by the piano Nikolayeva played on the evening of the concert. For whatever reason, however, some may wish to resist considering sounds to be reproducible---to be \emph{abstract} in this way. Nonetheless, even they can agree that in listening to the playback one hears some particulars sounds that, given a well-attuned stereo set, are likely to be heard as the sounds of a piano. %Needs work 

%OLD: resemble strikingly sounds that filled the concert hall on the evening of the performance. On such a picture, when we say that we hear the sounds of a piano, what we must mean is that we hear sounds that strikingly resemble the sounds of a piano.

% Argue: We can assume that we can hear the sound of a piano. We hear the piano when we listen to the recording. Two arguments: 
% [1] Generalisation argument: (softens up)
%If we can hear a piano as audience, then we we can hear a piano as listeners to recording; we can hear a piano as audience; we can hear a piano as listeners to recording. 
%[2] Phenomenological argument: (establishes)
%What will I do with the observation that we do not hear the speaker when we hear the sounds it reproduces. (We can distinguish between a speaker's producing and reproducing a sound, just as we can distinguish between someone saying something and someone rehearsing something.)

% NOTE (I need to note this: In fact I am developing three independently interesting arguments in this paper. I do think they should be developed together. First, there is the argument that we hear a piano when we listen to a recording. Second, there is the argument that this piano must be represented in experience (Local source-representationalism). Third, there is the argument that hearing all non-sound individuals must be represented in experience (Global source-representationalism). Spelling it out in terms of local and global versions helps to tie things together here.)

% SOUND (end)

%PIANO (fold)
\sect I now want to move on and argue that we hear a piano as well. 
%I should start by softening up
There is an argument from the descriptions we give, and the way we engage with recordings.
%Then I need a reductio of the converse: that we do not hear a piano
Assume that: If (we hear the sound of an F) and (no F is present to us), then it is not the case that we can hear an F. This is all too close a circle. We need additional elements. 

We experience the sound of an F as coming from D. [I NEED TO READ NUDDS AND O'CALLAGHAN AND DI BONA FOR THIS!] 

Imagine one is an apprentice professional tuner and has been given the job of tuning the cello of a string quartet that will be recorded for a CD release of Schubert's fourteenth string quartet. After the fact, one can surely listen to the recording and listen to the cello one has tuned. Some might just find it controversial that listening to a cello requires more than just hearing the sound of the cello.

Hearing the sound is sufficient, but it is not necessary. When one listens to a cello one has to hear it at some point, but not at all points. Listening to a cello only requires some hearing of a cello. I would beg the question if I would say: we are able hear a cello because we can listen to a cello if I would assume that listening to a cello requires being able to hear a cello. But that might be a discussion: does listening to something require some hearing of that thing, or does it only require hearing the sound of that thing. 

Can I move in the way of reductio? Assume that we do not hear a violin when we listen to the recording. Or did I want to use the broken speaker argument? What do we hear? The loudspeaker? Then I can argue that there are cases where we do hear the loudspeaker that clearly stand out as very different and as cases of `hearing something else'. That is one side of the coin. The alternative, we only hear sound. Can I then develop a continuity objection? We hear certain series of sounds as belonging together and we only do so because those sounds tell us about their sources. But does this mean that we hear those sources? 

A. Argue that we hear more than sound.
B. Argue that we hear either the piano or the loudspeaker
C. Argue that we do not hear the loudspeaker
Some might suggest that we do hear the loudspeaker when we think we hear the piano, the experience of the piano is illusory. But then, what in the ordinary case? If we assume there that we hear more than sound, and if we assume that there is no loudspeaker, then our e

I need an effective strategy to complete this part of the argument. We hear a sound, that we can assume. Why should we think we hear more than the sound? And if we hear more than the sound, then why think that the piano and the loudspeaker are the only options? [Think of the window-pane problem?] And even if they would be the only two options, why think that we do not hear the loudspeaker? Let us spell these arguments out separately. 

1. Why should we think we hear more than the sound? I know that Elvira di Bono thinks that we should think this because of timbre. 
[Perhaps I should take a much simpler case: a glass drops. You hear glass. You hear the sound of glass. ]

2. If we hear more than the sound, then why think that the piano and the loudspeaker are the only options? Let us assume that we hear a third thing. First we should say, if that third thing is not present, the argument still works. So it should be a third thing that is present. But, by hypothesis, the loudspeaker in the scenario is the only thing that actually produces sound. [We should remove the ceiling, and make it a corner or something.] Hence, an absent thing and a loudspeaker are the only options. As an absent thing we hear, a piano is the only thing consonant with the evidence we have. We can just assume the absent thing is a piano. 

3. Why think that we do not hear the loudspeaker? Let us assume we hear the loudspeaker and prove a contradiction. The differences in timbre of the sounds allows us to shift between attending to one thing and another, yet by hypothesis there is only one thing we hear and hence can attend to. 

Question: can I do anything to stop the illusion move? I can point to the observation that hearing the loudspeaker when you're listening to a piece of music is very distinctive. That experience if sufficient for hearing a loudspeaker. But does it follow that it is necessary? Someone could say that one hears the loudspeaker even if one thinks one hears a piano: this is an illusion theory. In the pictorial case, there is a clear argument: look at a picture, one is conscious of both a surface and something different. I should just make that move and stick my heels in the sand from that point: the experience of listening to a cracked speaker is my parallel case. This allows me to establish, I hope, that we do not hear the loudspeaker. 

% IMPORTANT TO PICK UP HERE: 
NOTE: I do not have to claim that we do not hear the loudspeaker! I have to claim that we do not only hear the sound and the loudspeaker. This is significant: It would be fine if we would hear and the sound and the speaker and the piano, for this would still get me to my conclusion. 

Hence, I should argue that we hear more than sound (a), and that we do not or do not only hear the speaker (b). Then, by hypothesis, the speaker and the sound are the only audible presences, and so we hear something absent. 

One thing we could say: Assume that we hear only sound of the piano and the speaker. Then now imagine that the speaker goes bust. Our experience changes. We could say that now we also hear the sound of the speaker. Perhaps I should do something with the perceived locations. There seems to be nothing special about the recording case, and so I already have an indiscriminability argument in play here? I could also leave this 

When we listen to a recording of a piece of music on a speaker set with a broken loudspeaker, we can suddenly start hearing the speaker. Speakers only function well if they are unheard. (There is a parallel with trope l'oeuil here, so be careful! Or, perhaps I can use this?) 

A loudspeaker is more like a picture's canvas than it is like a picture's paint. The sound the speaker produces is more like the paint that sits on the canvas. Both the speaker and the canvas are normally unseen. Some pictures are damaged, so that the canvas becomes visible, but this is disruptive. Similarly, when we hear the loudspeaker when we listen to a recording, this is disruptive. 

I could develop an argument on the basis of the location of the sources we hear. We hear where they are. Or, we hear some object as being some where. In the cracked speaker case, there is a clear conflict. We hear the speaker *there1*, but we hear the piano *there2*. 

There definitely is a discrepancy between the location of a loudspeaker and the experienced location of the sound it produces. But does this give us reason to think we do not or do not only hear the loudspeaker.
 

In any case, we have made some progress here: We should say that we hear more than just the sound, and we should say that we either hear glass or a loudspeaker, and we should say that we do not hear a loudspeaker. 


Imagine that one has tuned the piano strings that are struck by one of the hammers associated with a key on the piano. One can now listen to the recording of the concert and `hear out' for that particular piano one tuned. I believe that engaging in this activity successfully is certainly conceivable, and hence that one must be able to hear a piano when one listens to a recording. 
 
If we can `hear out for' a piano, then we can hear a piano; If we can `hear out for' a piano in some context, then we can hear a piano in that context; being a listener to recording is being in some context; If we can `hear out for' a piano as listeners to recording, then we can hear a piano as listeners to recording; we can `hear out for' a piano as listeners to recording; we can hear a piano as listeners to recording.

I can listen to a piano even if there is at some points where no sound of piano to be heard, or perhaps is no sound to be heard at all. Similarly, I can watch a bird even if at some points I can't see the colours of its feathers. This implies that listening to a piano is not reducible to hearing the sound of a piano, or hearing some sound as the sound of a piano. 



For this I have two arguments. The first makes use of a conditional, of which I assume the antecedent is true. If an audience at Nikolayeva's concert was able to hear her piano when they heard the sounds that filled the room that evening, then an audience listening to the recording of those sounds will be able to her piano when they hear the played-back sounds that fill their room on that occasion. % It may be that this argument is already too similar to what I want to develop later on. It would be good if I have a strong argument from phenomenology. I could use Heidegger? 

% I should take the 'only if there's a continuous interaction' reaction seriously.


The second argument has to do simply with our experience of listening to a recording. The argument makes use of the observation that we can `listen our for' some particular object in hearing a series of sounds. [Example]





 
 
Some may want to resist this on the basis of a conviction that not even the audience during the concert was able to hear the piano. They could either say that this is because all we ever hear are sounds, and piano is no sound. Or they could say, weaker, that all we ever hear are events, and a piano is no event. I will set aside the former view and simply assume that it is false that all we ever hear are sounds. To the latter view, I want to suggest that even if it is true that one could only hear events. the argument I develop can be reformulated without difficulty: wherever I speak of hearing a piano being played, these people are invited to read `hearing a piano's being played'.
% These are older elaborations of the above point
%As I said, I will simply assume that we hear a piano when we go to a concert hall and listen to a performance of a piano composition played on a piano. Those who disagree will be those who think that sources of sounds are never the objects of auditory perception. I will assume they are wrong. 
%Some might object and claim that they admit that sources of sounds are objects of perception, but that these sources are not object simpliciter, but always objects as and when they are involved in some event of sounding or noisy change. They will emphasise that we can only hear an anvil \emph{being struck}, Big Ben \emph{striking}, or a piano \emph{being played}. I think this is correct, but it would present a specification and not a denial of the claim I want to make. This is because hearing a piano being played is just one kind of hearing a piano. And so if an audience could only hear the piano being played on the evening of Nikolayeva's performance, they would still be able to hear a piano. 

%PIANO (end)

% REPRESENTATION (fold)
% Argue: If we hear the piano when we listen to the recording, then we perceive a piano that is not present to us. If we perceive something not present to us, the thing is represented in experience. If we hear the piano when we listen to the recording, then the piano is represented in experience when we listen to the recording. We hear the piano when we listen to the recording. The piano is represented in experience when we listen to the recording.
\sect Perceptual representation may be understood through a sufficient condition. If a perceiver perceives some object that is not present to the senses, then the object is represented in perception. Hence, we have a grasp on the phenomenon of perceptual representation, because we are at least able to identify some paradigmatic situations in which the phenomenon occurs. For present purposes, it is enough to think of such occurrences as being facilitated by a capacity we have for representing objects or properties in experience. 

Given this understanding of perceptual representation, we may say that our hearing a piano when we are listening to the recording counts as an instance of it. A piano is represented in our perception of our surroundings. Hence, we can conclude that at least a weak version of source-representationalism is correct: whenever we hear objects `in' or `through' hearing the sounds played back from a recording those objects are merely represented in experience. 

Some may be interested in enquiring into an explanation of this phenomenon. How is perceptual representation of this kind possible? Pointing to a manifestation of a representational capacity we have may not be entirely satisfactory. Perhaps, however, we should take seriously that the sound we hear plays an essential role in the exercise of that capacity. It may be that it is first and foremost that it is the sound we hear that represents a piano to us. If that is right, then we can further enquire what it is about that sound, as an object that is arguably just present in our experience, that makes it a candidate representation? 

At this point I do not want to move in this direction. Instead, I want to rehearse that we have seen that a local version of source-representationalism is vindicated, and in the next section argue that we in fact have reason to think that source-representationalism should be understood as a stronger, global thesis about our perception of the sources of sounds. 

% REPRESENTATION (end)

% GENERALISE (fold)
% Argue: If the piano is represented in experience when we listen to the recording, then the piano is represented in experience when we listen to it as an audience. [[WORK: This is because these experiences are subjectively indistinguishable, and belong to the same perceptual kind. ]]
\sect So far we have seen that in hearing a playback of a recording we can hear both the sound of the recorded instrument and the instrument itself. We also reached the conclusions that, although the sound may simply be present to the mind, the instrument must be represented if we hear it when we listen to a recording of its sound, given that this instrument is absent from the perceptual situation. However, these conclusions have been established in a discussion about hearing playbacks of recorded sounds. Can we generalise these findings to cases where an instrument is actually played to us? 

One thing we must see here is that listening to an instrument via a recording, and listen to it being played in real life may be indistinguishable for us. Think of those talented musicians in the bowels of London's Underground about which, at some distance, one is in doubt if they are actually playing their guitar, or just playing a recording of a guitar. More generally, we could  


It should be clear that, if the answer to this question were `no', then the actual presence the instrument makes onto the perceptual scene would make a difference to the kind of auditory experience we have. In other words, if only listening to recordings of a piano would be representational, then whether our hearing a piano is representational depends constitutively on whether a piano is present in the perceptual situation.




Earlier we conceived of a device that would generate auditory hallucinations of hearing a piano. When we have an auditory hallucination, we are in a state we could characterise as being similar to hearing a piano, without its being a hearing of a piano. There is another case which is comparable. We can imagine hearing a piano, and in doing so are in a state that may be characterised as like hearing a piano without its being a piano. A difference between the hallucination and the imagination, however, is that it is always obvious to us that our imagining hearing a piano is not actually a hearing of a piano. 

``Having the visual experience of water puts one in a position which is not neutral with respect to the actual environment as to whether blue water is present or not: that is how we have to characterise what our visual experience is like. Visualising the water puts you in a position of not being neutral with respect to the imagined situation. In visualising the expanse of water, one is not non-committal whether the imagined situation contains a blue expanse of water.'' (414)

``Having the visual experience of water puts one in a position which is not neutral with respect to the actual environment as to whether blue water is present or not: that is how we have to characterise what our visual experience is like. ''

As Martin explains ``Visualising the water puts you in a position of not being neutral with respect to the imagined situation. In visualising the expanse of water, one is not non-committal whether the imagined situation contains a blue expanse of water.'' (414)

In imagining hearing a piano, one equally is not entirely non-committal about what is contained in the imagined situation. One will take the situation to be one which contains the sound of a piano. However, in imagining hearing a piano one is non-committal whether the situation contains a piano or not. An explanation for this is that, as we saw, it is possible to hear a piano in situations that do not themselves contain a piano. What we can conclude from this is that the experience of hearing a piano, though constitutively dependent on the presence of the sound of a piano, is not constitutively dependent on the presence of a piano. 

(As the ancients saw well, the phenomenology of hearing voices is different. I am open to the suggestion that in imagining hearing Tom's voice we are committed to the imagined situation's containing Tom. That is why source-representationalism is a thesis about the sources of sounds, and not about the `sources of voices'. I am thankful to Margaret Hampson for discussion of this point.)

We can give another rationale for the lack of dependence on the presence of the piano. Martin explains that ``When one visualises an ocean like the Pacific, one imagines a blue expanse. Reflecting on what one’s act of visualising is like, one can attend only to the blue expanse that one visualises and nothing else. No surrogate or medium for the water or for the blue are evident to one in so imagining.''(413) But as we saw, in hearing a piano one can always attend to the sound of the piano as well. This suggests that the sound may function as a `surrogate' of `mediator' for the piano. It is not obvious that such an aural mediator is always evident to one, but as we saw it is always a possible object for attention. 


In this section I generalise the previous observation to all cases of sources representationalism. This turns the previous conclusion that some source hearing is representational into a stronger thesis: all source hearing is representational. I want to make use of a modified version of the argument from hallucination. 



My main move here will be to demonstrate that whether we hear the source or not does not constitutively depend on whether the source is present in the perceptual situation. That is, whether we hear a piano does not depend on whether a piano is present. We can proceed here by considering the case where we imagine hearing a piano. If imagine hearing a piano commits us to imagining a piano, then wether we imagine hearing a piano depends on whether we imagine a piano. If imagining O commits us to imagining P, then O constitutively depends on P. 




When we imagine seeing a piano, we are not taking there to be a piano before us, but we are non-neutral towards the scene we imagine: it contains a piano. 
When we imagine hearing a piano, we are not taking there to be a piano before us, but the non-neutrality of the scene is interesting: it contains the sound of a piano. 

I need to make it extremely obvious that in imagining hearing a piano all we do is imagine hearing the sound of a piano. What follows from this is that the experience of hearing a piano is neutral about the presence or absence of a piano. Hence, if the experience occurs as a representational experience in the absence of a piano, then if the experience occurs in the instrument’s presence it is representational as well. 



A. We could not have the fundamental kind of experience we have in R and C if the sound was not present in the perceptual situation. Hence, whether we have the fundamental kind of experience we have in R and C constitutively depends on whether the sound is present in the perceptual situation. In other words, what it is to have an experience of that kind in part depends on that particular sound's presence. [Point where the abstract/concrete issue becomes relevant.]

The representationalist about sound perception can deny this by making an assumption. We can have an experience of the same fundamental kind irrespective of the presence of the sound. 

B. 



In its canonical formulations, the argument from hallucination relies on a common kind assumption. This assumption requires that a hallucination and a veridical perception of a tree both have an experience of a tree as a common kind. Given that, whenever we are veridically perceiving a tree, we have an experience of the same kind as when we are hallucinating a tree,

Martin writes: ``As presented above, both sense-datum views and intentionalism assume that perceptual experiences form a common kind of mental state among cases of veridical perception, illusion and hallucination. The perceptual experience which one has when seeing a pig is of a kind which could have occurred were one not perceiving at all but having a visual hallucination indistinguishable from the sighting of the pig. On such an assumption, one may demand that whatever account one is to give of the experience one has when one veridically perceives, the same account must be applicable in some cases of hallucination.''

We can assume that we are perceiving in both C and R, that we are hearing the same (kind) of sound in C and R, and that we are hearing the same kind of instrument in C and R. We have established that in R the instrument is represented in experience. What reason would we have to think C is different, such that the instrument is not represented. The only difference to the case is the actual presence of the piano being played: does this make a significant difference to the phenomenology of experience? That is the question. If the answer is yes, we have reason to deny the common kind assumption for these cases. If the answer is no, we have reason to hold on to that assumption. 

Let us consider the visual case: here we have a similar situation. Do we have reason to think that the presence of objects makes a difference? The concern here is often explicated in terms of transparency. When we introspect in perception, we find the present objects we perceive. If we would not be perceiving, we would not be able to find those objects.

There are two discussions: SD on the one and NR and REP on the other; NR on the one and REP on the other. Which one am I interested in and why? 

%This is an argument! I can develop this on the back of Mike's 2002 argument. 
If we want to imagine hearing a piano, we imagine simply hearing the sound of a piano. This indicates that our experience is neutral to the presence or absence of the piano. 




Recording: one sits in a room, and the neighbours play a recording of Chopin's piano sonatas. Concert: one sites in a room, and the neighbours play Chopin's piano sonatas on their piano. Let us assume that in both R and C one hears the sound of a piano that comes through one's ceiling. Let us assume that in both R and C one hears a piano. And let us assume that R and C are subjectively indistinguishable. Someone might be in one of both without having any experiential grounds to rule out she is in either R or C. 

(What the room scenario does is remove the need to imagine no input from other sense modalities, which is not obviously possible.)

We assumed that hearing the sound of a piano and hearing a piano is present both in R and C: auditorily, that is their common factor. We can assume that in both cases one is perceptually related to a sound. The perceptual experience which one has in C is of a kind which occurs in R: one is veridically perceiving the sound of a piano. 

We can move by reductio. Given these assumptions, and given the assumption that in C a piano is present and in R a piano is absent, Does one's being in the kind of experiential state one is in depend constitutively on whether there is a piano upstairs or not? 
Compare: does it so depend on whether there is a sound of a piano or not? Yes, one could argue: compare the hallucination machine described earlier: one would not be hearing anything if that machine was active, even if the cases were subjectively indistinguishable. 
Reductio: given that it does so depend, the presence or absence of the piano must make a difference to the kind of state you are in. One could say that in C a piano is present and in R it is absent in experience. [[Getting close, but still not there!]]

Let me repeat what we have:
1. R and C are indistinguishable
2. R and C both involve hearing the sound of a piano
3. R and C both involve hearing a piano
4. R and C both involve the presence of the sound of a piano
5. C involves the presence of a piano
5. R does not involve the presence of a piano
Why would we say that the presence of the piano does not make any difference to your experience in C? I like Williamson's way of approa


This is a common formulation of the argument from hallucination (I follow Crane here): 

FIRST: It seems possible for someone to have an experience—a hallucination—which is subjectively indistinguishable from a genuine perception but where there is no mind-independent object being perceived.

This premise can be reformulated to be about the piano. It is possible for someone to have an experience which is subjectively indistinguishable from listening to a piano that is presently played in one's vicinity. 

SECOND: The perception and the subjectively indistinguishable hallucination are experiences of essentially the same kind.

This premise can be reformulated as a claim about the experience of the person listening to a recording on the one hand, and the experience of the person listening to the piano actually being played on the other. The claim would be that these are experiences of essentially the same kind. There will be controversy here about how to understand `essentially'. Perhaps I should rephrase this in terms of `fundamentally' and stick to Martin's formulation, as my argument is in part addressed to him (though only obliquely).

THIRD: Therefore it cannot be that the essence of the perception depends on the objects being experienced, since essentially the same kind of experience can occur in the absence of the objects.

Here we need to do a lot of interpretation. The reformulation needs to be that it cannot be that the essence of the experience of a member of the audience depends on the piano's being present, since essentially the same kind of experience can occur in the absence of the piano. Again, the issue will be about spelling out what we mean by `same kind' here.

CONCLUSION: Therefore the ordinary conception of perceptual experience—which treats experience as dependent on the mind-independent objects around us—cannot be correct.

I want to reach a different conclusion of course. I want to argue that the ordinary perceptual experience of hearing a piano does not depend on the presence of the piano (but only on the presence of a sound of a piano). 

It will take me too long to comment on the argument from hallucination in such detail, and so I should first run my own version and only afterwards mention that it in fact is an application of the argument from hallucination. This will be fairly obvious anyway.

There is something funny going on, and I want to pause and describe this. We can make two moves: 
When we hear a sound and its source, then an experience of that kind will be subjectively indistinguishable from hearing  a recording of that sound. 
When we hear a sound and its source, then an experience of that kind will be subjectively indistinguishable from hearing a the source actually produce the sound.
I recall from discussions with Roberta that there is an asymmetry that the naive realist exploits in resisting the argument from hallucination. This has something to do with phenomenal character. That might be a way in to make the argument work for source representationalism. 

Someone who disagrees with me on this point, what will they think? They will agree that hearing the piano when we listen to a recording is representational. But they will claim that when the piano is played to us actually, we will be able to hear the piano in a non-representational way. So, by hypothesis the piano is present and we hear the same (or a similar) sound. The question is, does this add something to the phenomenology of the experience? 

Reflect on the classical argument again. Here, the hallucination and the perception are said to have the same phenomenology. Yet, this is only true if we assume that there is a common element involved in both. Yet, someone might deny that there is a common element involved, because someone may deny that there is anything we become aware of in a hallucination. A hallucination may be defined just as an experience that is subjectively indistinguishable from a veridical perception. 

The sound case is interesting, because there we have a common element of more or less the kind the sense datum theorist at least dreamt of. Should this be my main focus? Should I claim that the `common kind' assumption must be made for the comparison I am interested in? This leads to three questions. 

1. Can sounds be a common factor of the relevant sort?
2. Does assuming the common factor claim validate the argument from hallucination? 
3. Does assuming the common factor claim validate the argument for source representationalism? 

% GENERALISE (end)

% EVALUATE (fold)
% The experience of tinnitus does not similarly move us to accept that hearing sounds is representational. But we have seen that we should accept source representationalism: that hearing sources of sounds is representational. 
\sect

For perception to be representational, we earlier noted that it is sufficient if we have an object of perception that is absent from the perceptual scene. Now we see that this is not a necessary condition. Often the objects we hear are present to us, for instance because we see them (assuming naive realism). What is necessary for an aspect of experience to be representational is that the object could be absent. This entails that some object can both be presented and represented in experience. This is interesting to develop.

Emphasise that this argument has no force in the debate between naive realists and representationalists about experience. I find it plausible to think that sounds are not represented in experience, but someone might wish to try to defend the thesis that they are, because perceptual experience as such is representational. If such an argument would succeed, then source representationalism is still true---though I wonder whether the representationalist will use `representation' in exactly the sense we have seen sources are represented in experience. 

For further research, It is a good question how we should approach the hallucinatory case. If we have a hallucination of hearing a piano, should we still say that a piano is represented in experience, or should we say that we have a hallucination of a piano's being represented in experience? 

What I aimed to do in this paper is make plausible at least the following conditional. If one accepts that it is possible to hear more than sounds alone, then those non-sound objects one hears must be represented in experience. 


% EVALUATE (end)

% CONCLUSION (fold)
Here is the main argument again. `(C1)' means that these are conclusions of further arguments I need to offer.

\begin{itemize}
	\item We hear the sound of the piano when we listen to a recording of a piano being played (C1)
	\item We hear a piano when we listen to a recording of a piano being played (C2)
	\item If we hear a piano when we hear€ a recording of a piano being played, then a piano is represented in perception (C3)
	\item If piano is represented in perception, then the perception of a piano does not depend on the presence of a piano (C4) 
	\item If a perception of a piano does not depend on the presence of a piano, then a piano is represented in experience (C5)
	\item If we hear an actual piano being played we have an experience of fundamentally the same kind as listening to a recording of a piano being played (C6)
	\item If we hear an actual piano being played, then the perception of a piano does not depend on the presence of a piano (C7: From C4 and C6)
	\item If we hear an actual piano being played, then a piano is represented in experience (C8: From C5 and C7)
\end{itemize}
% CONCLUSION (end)




%Bibliography: \standardbib just loads a regular bibliography. All files have been loaded in the preamble.
\printbibliography
\end{document}

