\documentclass[sloppy, journal, git, anonymise]{humapap}
%\usepackage[]{org-preamble-pdflatex}
%\input{vc}



\thetitle {Source representationalism}

%\author{}

\myauthor{Maarten Steenhagen}
\myemail{} 
\theaffiliation{}

\thehtitle{the header} % Title in the header (no caps is fine)
\mydescription{~} % Hovers over the title. Can be left blank (~)
\thejournal{the journal} % Left corner subsequent pages
\theyear{2014}
\thanks{}
\remote{https://bitbucket.org/msteenhagen/humapap/commits/}
%\author{Maarten Steenhagen (University College London)}

\date{}

\begin{document}  
\setkeys{Gin}{width=1\textwidth} 
\documenttitle	
%\setromanfont[Mapping=tex-text,Numbers=OldStyle]{Minion Pro} 
%\setsansfont[Mapping=tex-text]{Minion Pro} 
%\setmonofont[Mapping=tex-text,Scale=0.8]{Pragmata}


%\published{January 2014. Incomplete Draft. Please do not cite without permission.}


% 
\begin{abstract}
\noindent Source representationalism is the thesis that the sources of the sounds
we hear, whenever they are heard, are represented in experience.
Starting from arguments about listening to recordings, I develop a
defence of this thesis. I show that listening to recordings enables us
to hear sound sources representationally. I then argue that, given the
structure and character of auditory perception, hearing a source when it
is actually producing sounds is equally representational. This
establishes source representationalism as a global thesis about auditory
perception. This has important implications for our understanding of
perceptual representation more generally.
\end{abstract}

\noindent Source representationalists make a claim about auditory perception. They
maintain that the sources of the sounds we hear are represented in
perceptual experience, whenever they are heard. Source
representationalists may accept that, when we hear a piano being played,
our auditory experience consists in an auditory presentation of the
sound of a piano. Yet they deny that hearing a piano can consist in an
auditory presentation of that piano itself. In this paper I show why we
should accept this claim.

\subparagraph{§1.}\label{section}

You listen to a stereo recording of Keith Jarrett's \emph{The Köln
Concert}, a piano improvisation performed on the 24th of January 1975 in
Cologne. Its actual audience was seated in a celebrated opera house
designed by Wilhelm Riphahn. You, however, are in London, in an almost
empty, quiet study. It's 2014. Your only company is the fauteuil you sit
in, and a high fidelity stereo set. The quality of the recording and the
calibration of the loudspeakers are superb. You press play, well aware
that you're in for a treat. Leaning back, eyes closed, what can you
hear?

\subparagraph{§2.}\label{section-1}

First of all, you can hear the sound of a piano. This particular sound
becomes perceptible as soon as it fills the room when playback starts.
Closely attending to Jarrett's performance, the sound may subtly
decompose into individual sounds of piano keys played by skilled hands.
Let's assume of all these audible sounds that they are instances of the
sound of a piano.

We may contrast this experience with one in which a device is used that
causes us to have auditory hallucinations. Playing back a recording on
such a device would result in undergoing an experience that in some
subjectively salient respects resembles hearing Jarrett's solo
improvisations, while in others it resembles tinnitus. But enjoying a
piano recording by means of such a device would however not involve
hearing the sound of a piano.

Listening to an ordinary playback of the album release of Jarrett's
\emph{Köln Concert} is no hallucination. Your experience not merely
resembles hearing sounds of a piano trickling into the room; it is such
a hearing. It is a veridical perception of playful sounds whose
perceived characteristics make up the phenomenal character of your
auditory experience.

This could trigger a question. What is meant by saying that in your
empty study you can hear the sound of a piano? Can you genuinely hear
the sound of a piano in the absence of any such instrument? Some argue
that it is of the nature of sounds that they are reproducible (Martin
2012). If they are right, then your speakers could reproduce the
particular sounds that were produced by a Bösendorfer piano in January
1975.

Some may want to resist attributing to sounds an inherently reproducible
and abstract nature. However, even they could agree that, given a
well-calibrated stereo set, a playback can comprise sounds qualitatively
identical to, and hence indistinguishable from, the sounds Jarrett's
piano produced. Playing back a recording may allow you to hear sounds
that have every audible characteristic in common with the original,
recorded sounds.

\subparagraph{§3.}\label{section-2}

In general, we can hear more than sounds alone. J.O. Urmson writes,

\begin{quote}
Clearly one may see, hear, feel, smell, and taste physical objects like
motor cars and apples; it requires ingenious stage setting to make `I
hear an apple' or `I taste a motor car' sound natural \ldots{} but one
may certainly hear a motor car or taste an apple (Urmson 1968, 117).
\end{quote}

Just as we can see a magpie, or taste vinegar in our salad, we can hear
things such as people in the corridor, aeroplanes overhead, and a pub
fight down the street. We can hear their sounds as well. But there is no
good reason to adopt the restrictive view, advanced by Berkeley's
mouthpiece Philonous, that sounds are all we hear (Berkeley 1954).

This observation is not merely a report of common sense; it has a
compelling basis in the phenomenology of audition as well. Sounds and
sources can be distinct objects of attention, and on occasions may even
compete for it. Recently, a number of authors have argued that our
ability to attend to things that are not sounds reveals that the objects
of audition include more than sounds alone (See: O'Callaghan and Nudds
2009).

Imagine that at night you hear a sound coming from your living room.
Reluctant to get out of bed, you first listen more carefully to what
goes on. You realise that it's the cat who managed to open the kitchen
door again. You don't pay much heed to the sounds you hear. No doubt you
could set yourself to attend to them, but you have better things to do:
tracking a feline threat to designer furniture. Attentively you listen
to the cat; you may listen to it scratching the sofa, or, if you're
unlucky, hear it throwing over a vase (and, inevitably, hear the vase
breaking).

We can attend to sounds, but we can often also attend to, and hence
listen to, what a sound is of. Against Berkeley, not only are we happy
to characterise the experience described above as one of `hearing a
cat', but moreover, the fact that we can distinguish between the
different acts of attending proves that we can hear more than sounds
alone.

This means significantly more than that we are able to hear \emph{that}
a sound is of a cat, or \emph{that} the cat is in the room. Such
epistemic hearing of facts about sounds or pets does of course occur,
but the point is different: listening attentively both to a sound and to
a cat is only possible if both sound and cat can be heard. Just as
Urmson's motor car, your cat, as the source of a sound you hear, can be
an object of audition.

If this is right, then by the same reasoning we may conclude that in
listening to a recording we can hear a sound's source. Consider an
example. To find out what had happened in the Boeing 737 that crashed
near Pittsburgh in September 1994, a team of audio forensic
investigators played back the cockpit voice recorder that was retrieved
after the accident. The team was particularly interested in a
malfunctioning of the plane's rudders. Hence, it is unlikely that they
listened merely to the sounds reproduced by their playback equipment.
Indeed, they found out that the rudder's eventual jamming was not due to
mechanical obstruction, but because of inept operation of controls. Such
a finding was only possible by listening attentively to two sources
audible in the recording: the plane's rudder and the crew's manipulation
of switches and dials on the dashboard.

This suggests that also in listening to a recording it is possible to
turn one's attention to a sound's source, and hear it. Whether one can
engage in this mode of attentive listening to sources depends on whether
one hears the right sounds, not on whether it is a recording.

However, someone could object that the case of hearing a cat at night
and hearing the recording of a plane's engine are different. Where in
the former case there actually is a cat producing those sounds, in the
latter a loudspeaker is directly responsible for the audible sound.
Hence, the objection would run, the fact that one can attend to
something beyond the mere sound of a recording suggests just that we can
hear the loudspeaker that reproduces it.

This objection is unconvincing. As every audiophile will attest,
loudspeakers are inaudible if they function well. Imagine that, while
listening to the recording of Jarrett's play, an annoying tearing noise
makes itself heard. Your loudspeaker is bust, and its sound now
interferes with the piano's melodies; such a tearing will stand out in
experience as not belonging to the recording itself. This is what
hearing a loudspeaker is like. Understanding what it would be to hear a
loudspeaker requires consideration of cases where the device has a
noticeable defect. It should be clear that, instead of setting a
standard for listening to recorded sounds, such an experience just
signals that equipment needs to be replaced.

\subparagraph{§4.}\label{section-3}

We wondered what one could hear when one plays a recording of Jarrett's
improvisations. We may now answer that one can hear not only sounds that
have every audible characteristic in common with the original sounds of
Jarrett's performance, but also a Bösendorfer piano---the source of
those sounds.

\subparagraph{§5.}\label{section-4}

The following is a sufficient condition for perceptual representation:

\begin{quote}
\textbf{Perceptual representation} If a perceiver perceives some object
that is not present to the senses, then the object is represented in
perception.
\end{quote}

This condition allows us to identify typical occurrences of perceptual
representation. In particular, it allows us to say that if we merely
play back a recording of a piano, then any hearing of a piano counts as
an instance of perceptual representation. This is because we would hear
the instrument while it is not present to the senses. And this in turn
suggests that hearing a sound with the right characteristics suffices
for being able to hear a piano representationally.

Assuming that there is nothing peculiar about pianos, we may now accept
a restricted version of source representationalism. Whenever we hear a
source while merely playing back a sound recording, the source is
represented in experience, because we hear it while it is not present to
the senses.

This raises a question. By what mechanism is perceptual representation
of this kind possible? Merely suggesting we possess a capacity for
representational perception may not be entirely satisfactory. Yet, we
can say more. The heard sound plays a determining role in the exercise
of this capacity.

Just as other perceptual capacities, the capacity for perceptual
representation is only exercised in response to something---it is a
\emph{reactive} capacity (cf. Kalderon forthcoming). Sound figures as an
essential cog in the machinery of representation in auditory perception.
This is because, in general, what we can hear is at least in part
determined by the audible character of the sounds we perceive. And if
the sound reproduced by our stereo set did not have audible
characteristics of the right kind, hearing that sound would not enable
us to hear a piano at all. We hear a piano because we hear a sound with
the right characteristics.

This suggests that it is the heard sound that represents a piano to us.
Naturally, this leads us to ask what it is about that sound, as an
auditory object, that lets it take on such a representative attire. Here
I will not address this question. Whatever account we give of the
characteristics of sounds, or of perceptual representations more
generally, if the previous discussion is along the right lines, we do
possess a psychological capacity to make use of them. Hearing sounds
with the right character puts one in a position to exercise this
capacity.

In what follows I show that this understanding of auditory
representation implies that source representationalism should be
accepted as a stronger, global thesis about auditory perception.

\subparagraph{§6.}\label{section-5}

We observed that we can hear both the sound of a piano and a piano
itself in listening to a recording of Jarrett's \emph{Köln Concert}. We
reasoned that in such circumstances, although the sound may be present
to the mind, the piano we hear is represented in auditory experience.
Hearing sounds with the audible characteristics of the sound of a
Bösendorfer piano enabled us to hear such an instrument in its absence.
Now, is the experience of hearing a piano when such an instrument is
actually being played to us also one of hearing a piano
representationally?

Think of those talented musicians in the bowels of London's Underground.
At some distance, one may be in doubt whether they are actually
strumming their guitar or are relying on a covertly playing recording.
In such a state of doubt, reflecting on what one hears may not be
sufficient to distinguish between the two possible scenarios, because
for any \emph{original} experience of a sound actually produced by the
source one hears, there is a conceivable, indiscriminable
\emph{recording-counterpart} in which we hear that source through a
recording instead. Both auditory experiences may very well have the same
phenomenal character.

Hence, and more generally, of any original auditory experience and its
recording-counterpart we may ask: Do they differ in phenomenal
character? If they do not differ, then we may conclude that if a piano
is represented in the recording-counterpart experience (which is true,
it turns out), then a piano is equally represented when we hear such an
instrument when it is actually being played to us. This is because two
experiences with the same phenomenal character will be experiences with
the same representational properties.

There is good reason to think that for every original experience there
is a recording-counterpart that does not differ in phenomenal character.
This is not because they are subjectively indistinguishable; subjective
indistinguishability is generally no adequate criterion of sameness.
Instead, the reason is to be found in the structural role the audible
characteristics of sounds play in determining the phenomenal character
of our hearing of sound sources.

As we saw, we can only hear a sound source if we hear a sound with the
right characteristics. When we hear a piano we do so because we
experience the specific audible characteristics of sounds that reach our
ears. In most cases, what it is for us to hear those characteristics
depends on what it is for us to hear a source of some kind. What it is
to hear a sound with the characteristics of the sound of an airplane
rudder, for instance, depends on what it is to hear parts, properties
and materials of such a rudder itself. In other words, hearing those
specific characteristics just is what it is to hear a rudder. This is
confirmed by the fact that when we are asked to describe such audible
characteristics of sound, we are typically bound to describing the
sources and their properties they allow us to hear (Nudds 2010, 284).

So is the experience of hearing a piano when such an instrument is
actually being played to us one of hearing a piano representationally?
Yes it is. If hearing a Bösendorfer piano consists in hearing the
specific audible characteristics of the sound of a Bösendorfer, then
having that experience does not depend on whether we hear sounds
actually produced by such a piano or a recording of them. This is
because playing back a recording may allow us to hear sounds that have
every audible characteristic in common with the original sounds. This
implies that our experience in hearing the recording exemplifies what it
is to hear a sound source. If we hear a source when we listen to a
recording, then the heard source is represented in experience. The same
experience occurs when we hear a piano that is actually being played to
us. Therefore, also this experience will be one of hearing a piano
representationally.

\subparagraph{§7.}\label{section-6}

Source representationalism is the view according to which the sources of
the sounds we hear---cats, pianos, aeroplanes---are, whenever we hear
them, represented in experience. The analysis of listening to things
through recordings, and the established conceivable sameness of
phenomenal character of hearing recorded sounds and hearing sounds that
are actually produced, demonstrates the correctness of the source
representationalist claim. Whenever we hear a piano or some other sound
source, this object of audition is represented in experience.

The correctness of the source representationalist claim has broader
significance. The observation that sounds are an indispensable element
in the workings of perceptual representation in the auditory
modality---that they are the objects of perception that fulfil a
representational function---suggests a broader programme. It indicates
that a complete picture of the scope of perceptual experience must
consider how, in more perceptual modalities, representations figure in
our perception of the world.

\subsection*{References}

\setlength{\parindent}{-0.2in} \setlength{\leftskip}{0.2in}
\setlength{\parskip}{8pt} \vspace*{-0.2in} \noindent

Berkeley, George. 1954. \emph{Three dialogues between hylas and
philonous}. New York: The Liberal Arts Press.

Kalderon, Mark Eli. forthcoming. \emph{Form without matter: Empedocles
and aristotle on color perception}. Oxford: Oxford University Press.

Martin, M.G.F. 2012. ``Sounds and images.'' \emph{British Society of
Aesthetics} 52(4): 331--351.

Nudds, Matthew. 2010. ``What sounds are.'' In \emph{Oxford studies in
metaphysics 5}, ed. Dean W. Zimmerman. Oxford: Oxford University Press,
p. 279--302.

O'Callaghan, Casey, and Matthew Nudds, eds. 2009. \emph{Sounds and
perception}. Oxford: Oxford University Press.

Urmson, J. O. 1968. ``The objects of the five senses.''
\emph{Proceedings of the British Academy} 54: 117--131.

\end{document}