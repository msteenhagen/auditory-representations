% 
%  thoughtsnotthought.tex
%  
%  Created by Maarten Steenhagen on 2013-07-01.
%  Copyright 2013 Maarten Steenhagen. All rights reserved.
% 
\documentclass[sloppy, journal, git, bytitle, dodraft]{humapap}
\usepackage{soul}
\myauthor{Maarten Steenhagen}
\myemail{\\m.steenhagen.09@ucl.ac.uk\\} % Will be used on first page, loaded in author.tex
\mydraft{Draft version}
\theaffiliation{}
\thetitle {following audition's lead} % Fist page (no caps is fine)
%\mysubtitle{a second title} % First page
%\mydraft{Draft - do not cite}
\thehtitle{the header} % Title in the header (no caps is fine)
\mydescription{Draft version} % Hovers over the title. Can be left blank (~)
\thejournal{the journal} % Left corner subsequent pages
\theyear{2013}
\thanks{}
\remote{https://bitbucket.org/msteenhagen/humapap/commits/}
\begin{document}
\documenttitle

\begin{abstract}That some objects are, as the naive realist has it, presented in perception does not rule out that some objects of perception are represented. I show that at least some plausible versions of naive realism to accept that audition essentially tends towards representational perception. The argument for this is that,  on the one hand, our capacity for auditory perception is for the sake of hearing the sources of sounds, while on the other those sources, if they are not sounds, cannot be constituents of auditory perception. This implies that audition essentially has a tendency to afford perception of objects that need not be constituents of experience, which in turns suggests that such perceptions defy the non-representationalist analysis offered by the naive realist and, hence, must be represented in experience.\end{abstract}
	
	%Don't forget: represented = perceived & not presented. (Instead of presence in absence, it is perception in absence.)

% INTRODUCTION (fold)
\dropcap{H}{ow does the world} become manifest to us in sense perception? P. F. Strawson suggests that it is undeniable that sense perception passes itself off as ``an immediate consciousness of the existence of things outside us'' (1979:99). To us things seem to be immediately present when we perceive. One might think, perhaps naively, that things could actually be like that---that sensory perception could actually be an immediate kind of consciousness. 

Developing this line, some have suggested that we can be and often are aware of various objects that, quite simply, are \emph{presented} in sense perception. That is what all sense perception is, fundamentally. For an object to be presented in sense perception is for it to be a \emph{constituent} of our perceptual experience, where the concept of a constituent of perceptual experience is introduced to capture the idea that some objects in our environment, in part because of the sensible qualities of these objects, in part because of the reactive nature of our sensory capacities, can determine directly the character of our perceptual experience (see e.g. Martin 2004, Travis 2004, Kalderon 2011).   

However, in maintaining that  sense perception is, fundamentally, an experience determined by how objects in the world are, the defender of such a naively realist position is committed to a further claim. They must claim as well that at least to some extent sense perception is non-representational. They must, because we may assume that if sense perception is wholly representational, so that objects in the world would merely be represented in perception, then the way these objects are---their natures and qualities---could not determine the character of our perception directly, but only indirectly, by being inscribed in the representational content of the perceptual representation. 

%PLAN
That some objects of perception are presented in experience does not rule out that some objects of perception are represented. Even if one concedes that sense perception is at least to some extent `presentational', and so non-representational, this leaves open the possibility that there still are respects in which such perception is representational.  In this paper I argue that even the naive realist might be compelled to make room for perceptual representation. More specifically, I suggest that audition at least typically involves a species of representational perception: representational hearing. The argument for this is that,  on the one hand, our capacity for auditory perception is a capacity for hearing the sources of sounds, while on the other hand only the sounds we hear can be constituents of auditory experience. If that is accepted, it suggests that audition by its very nature has a tendency to afford perception of objects that need not be constituents of experience. This suggests that audition to that extent defies the non-representationalist analysis offered by the naive realist and, hence, must be a species of representational perception.  
%END SECTION (end)

% The idea of constituents, and the argument for premise (2) (fold)
\sect Let me first give a partial characterisation of the naive realist understanding of the nature of sense perception. For present purposes, it is useful to understand naive realism as primarily the denial of strong representationalism about perception. This is often the way the view is presented and discussed. 

We perceive something only if it, either directly or indirectly, makes a difference to the character of our experience.  Strong representationalism about perception, at its core, claims that the character of our perceptual experience, what features and objects we become aware of in perceptual experience, is always fundamentally determined by the representational content associated with that experience, and not by the reality that content is about. As Adrian Haddock and Fiona Macpherson put it, all perceptions ``represent the world to be a certain way, and the world may, or may not, be that way.''\autocite[p. 14]{haddock2008aa}

Naive realism denies this claim. What we become aware of in sense perception, according to the naive realist, can fundamentally be determined by non-representational properties, simply by being a constituent of the perception. Something is a constituent of perceptual experience only if it directly determines or shapes the character of perceptual experience. The naive realist the metaphysical thesis that sense perception fundamentally has such constituents, in this sense. As Mark Kalderon writes,  
\begin{quote}
since my perception is constitutively linked to the tomato, the tomato, itself dappled in sunlight and shadow and partially obscuring the view of the chapel, shapes the contours of my sensory consciousness by being present in that consciousness.
\end{quote}
The metaphor of shaping the contours of sensory consciousness Kalderon uses suggests that if a veridical perception at some moment is merely of a constituent tomato, then the way things strike you in that perceptual episode---how things appear to you---depends entirely on the tomato itself and its sensible qualities. 

The metaphysical thesis at the core of any naive realist position will be that in sense perception something is able constitutively to shape or determine what our experience, at least in part, is like. This feature of perceptual experience will be relevant for an understanding of any instance of sense experience, because it is essential to it. If the metaphysical thesis is correct, it proves that it is at least possible for objects in your surroundings to become directly manifest to you. 

The strongest version of naive realism will maintain that this in fact exhausts the character of sense perception in its entirety, so that only the constituents of experience can determine the character of perceptual experience. But, more neutrally, we may say that according to all versions of naive realism, if sense perception of some object is not representational, then if one can perceive the object, then the object is a possible constituent of sense experience. If you can be said genuinely to see the tomato in a non-representational way, then the naive realist must claim that the tomato is a constituent of your experience. 

This puts us in a position clearly to see why this metaphysical thesis about sense experience advanced by the naive realist implies that at least to some extent all sense experience is not representational. This is because, if experience would be wholly representational, then the way objects in the world are could not directly determine the character of our experience. We may assume this because representation essentially allows for misrepresentation, as reflected in Haddock and Macpherson's earlier claim that the way things are represented in perception leaves it open whether things are, or are not, that way. In other words, a globally representationalist analysis of perception will always leave it undecided whether a  perceptual state with specific representational properties is in any respect veridical or not. This implies that to the extent our sense experience of an object is representational, it can at best be the way the object is represented that determines the experience's character. The way an object is represented is unmistakably a property of the representation, and not a property of the represented object itself. (Unless, of course, objects would represent themselves, but this is not what the representationalist has in mind) As we could put it, if an object is merely represented in experience then it can only affect by being inscribed in the content of representation in some more or less roundabout way, that is, it can only affect us indirectly. 

The naive realist maintains that in sense perception some object affects experience directly, and for this to be possible such an object must be a constituent of experience and cannot merely be represented. In other words, naive realism about our perception of \emph{x} can at least in part be understood as the denial of representationalism about our perception of  \emph{x}.
%END SECTION (end)

% The structural difference (fold)
\sect It is important to observe that the way constituents of experience could determine how things appear to you in experience varies across modalities. Most developments of naive realism have focused on vision, which is comparatively straightforward, yet the difference between vision and audition is striking and important to acknowledge. 

Imagine yourself loitering in the centre of a crowded space, an airport terminal, say. You see the passengers, plastic seats and coffee bars at varying distances. You hear the noise of rolling luggage, you hear voices and the aeroplane outside on the runway. All these hearings count as perceptions of those objects. Yet on reflection, it becomes clear that the way the terminal and its occupants can figure in visual experience is relevantly different from how they figure in auditory experience.

To see why this is, consider vision first. Your capacity for sight enables you to see, say, the travel-prepped people that stand before you in the bleakly illuminated departure hall. They struggle to hold both their trolleys and take-away beverages, they gesture at one another and point at the departure sign. How these people appear to you visually depends centrally on their visible features---the colour of their clothes, the movement of their hands, the expression on their faces. In order for these people to impact on your conscious perception in the way they do all that is required is a proper illumination of the scene, and a proper functioning of your sense organs; crucially, no further other \emph{objects} need to be perceived for that dependence to be realised.

Audition is different. You may claim to be able to hear those people, and so perceive them in that way, but this could only be possible if you would hear also something else, something distinct from them. This is because to hear those people you must also hear the sound they produce. This should be obvious. We might say that if the touristic group would not produce any sound, or if the sound they produce would for some reason inaudible to us, we simply would not be able to hear them . Sounds, in that sense, can be conceived of as a \emph{primary} object of perception, as a special kind of object that must be  heard if anything is heard at all \autocite[p. 118]{urmson1968aa}. 

The sounds we hear are not integral parts or qualities of things like people or jet engines. We think of sound as something that can fill a room or bounce of the walls of a canyon---feats of spatial existence qualities cannot rival. In the way we report the character of our experience, we may distinguish hearing ordinary material objects from hearing sounds; although we do not allow for the possibility of hearing a material object without hearing a sound, we allow the for the converse possibility. As Urmson observes, ``one cannot claim to hear a motor car but deny that one hears a sound, whereas one may be unwilling to commit oneself further than that one hears a sound'' \autocite[p. 118]{urmson1968aa}.

As perceptible individuals, sounds may be thought of as products of particular people, jet engines and the like. Sounds are produced by producers like a camera produces a flash or a flung stone the ripples in a pond. This suggests sounds have a degree of independence of both their producers and events of production. Being a product, a sound can be present in one room, while its producer is in the room next door or on the other end of a telephone line. 

This explains why we may think of sounds as particular objects of perception in their own right, objects that are necessary accompaniments of any act of audition. Sounds may be conceived of as members of a special kind of primary objects of hearing, a kind of object vision lacks. We do not conceive of the visual world as containing a special class of objects, a class distinct from the material world we can see, through which we come to see the material world \autocite[cf.][p. 334]{martin2012aa}. Sight simply enables us to lay eyes on the surfaces of material bodies, panelling and aluminium finish. This makes for an important structural difference between vision and audition. 

% It is plausible to assume that , when we hear a sound, it is able  to determine directly the character of our perceptual experience, which makes  sounds themselves constituents of that experience. If sounds are constituents of perceptual experience, and if our hearing those people depends on hearing the sounds they produce, then the character of the auditory perception of those people will at least in part depend on the the audible features of the sounds you hear. This means that if you have a veridical auditory perception merely of those people, then the way things strike you in experience---how things appear to you---can not depend entirely on those people itself and their sensible qualities. It must at least in part depend on the sensible qualities of the sounds you hear as well. 
%END SECTION (end)

% WE CAN HEAR SOURCES: Argument for premise (1) (fold)
\sect If one accepts that naive realism implies the denial of strong representationalism about perception, and if one agrees with the structural difference between visual and auditory perception as developed above, then the following argument could compel one to accept that sense perception in the auditory modality is at least in part representational. This is not an argument against naive realism as such, but it does seem to present a puzzle about the potential plurality of perceptual kinds, and a challenge for naive realism's strongest versions. 
\begin{enumerate}
\subsubsection*{Argument for `source representationalism'}
\item If hearing sources of sounds is not representational, then, if one can hear sources of sounds, then sources of sounds can be possible constituents of auditory perception.
\item One can hear sources of sounds.
\item It is not the case that sources of sounds can be possible constituents of auditory perception.
\item Therefore, hearing sources of sounds is  representational.
\end{enumerate}

The argument is valid, at least if we make the seemingly innocent assumption that if it is not the case that hearing sources of sounds is not representational, then hearing sources of sounds is representational. Its conclusion, that hearing sources of sounds is  representational, contradicts the strong representationalist thesis that the character of sense perception is exhaustively determined by the constituents of the perceptual experience. This is because we can only be said to perceive something if it, either directly or indirectly, makes a difference to the character of our experience. Further, as I will show, each of its premises are defensible on the naive realist's own terms. 

The first premise of the argument is entailed by the naive realist's opposition representationalism about perception. The naive realist seems committed to this complex conditional, because naive realism about our perception of \emph{x} may be understood at least in part as the denial of representationalism about our perception of  \emph{x}.  At least for the moment, I will assume  that the naive realist accepts the first premise. 

If the first premise is true, then it follows that someone who wants to reject representationalism about source perception, and so defend a specific local form of naive realism, is committed not to accept both the second and the third premise---that is, they must at least give up one of these premises. 

What I want to show in what follows is that doing so will not be uncontroversial. This is because, on the one hand, the rejection of the idea that we can hear the sources of sounds will by many be regarded untenable in light of both phenomenological and scientific considerations, and, on the other hand, the idea that sources could be constituents of auditory perception seems to put the notion of a constituent of perception in a bad light.  In what follows I will explain these two difficulties in turn.

\sect As is well known, the question whether we can actually hear sources of sounds has been subject of  controversy. Some have been tempted to think that the structural difference between vision and audition, and especially the status of sounds as a primary object of hearing, implies that, whereas we can see a variety of material and immaterial objects, such as people, rainbows and jet engines, all we can ever come to hear are the sounds they might produce. The obvious fact that audition offers us knowledge of these sources must be explained in terms of a quick or unconscious inference, or a kind of `epistemic' hearing.  

Berkeley famously maintained that all we can hear is sound, because it is only sound that possesses the kinds of sensible qualities hearing is sensitive to. This is a strong claim, one that is in tension with the way auditory experience strikes us, and the kinds of descriptions of auditory perception we may offer with complete linguistic propriety \autocite[p. 117]{urmson1968aa}. More often than not we are very willing to self-ascribe the perceptual experience of hearing the sources of sounds \autocite{broad1965aa}.

More significantly, the austere Berkelean denial of source perception seems to rest on a mistake, because it contradicts a very plausible thesis about the nature of our capacity for hearing. Matthew Nudds has argued that our ordinary understanding of the scope of auditory perception is in the right. He observes that
\begin{quote}
We are typically very good at identifying both the kind of event---footsteps, a door opening---and the kind of object---a person, a door---that produced the sounds we hear, and we are often able to perceive features of that object or event—the hardness of the object, the force with which it was struck, its location, whether it was in an enclosed space, and even its approximate size and shape. Most everyday hearing is of this kind: we attend to the apparent sources of the sounds we hear and listen to the things going on around us---to the objects and events that produce sounds (`sound sources' for short). \autocite[pp. 283-84]{nudds2010aa}
\end{quote}

Nudds' key argument is that we are sensitive to the sounds in our environment only because these sounds provide us perceptual access to sound sources. What sound we hear is determined by the way our auditory system at a sub-personal level groups a complex array of frequency components. The principles that constrain such groupings, principles that explain why we hear the sounds we hear, are all aimed at retrieving source information. As Nudds writes, ``perceiving the sources of sounds is what auditory perception is for'' \autocite[p. 284]{nudds2010aa}.

That our capacity to hear is first and foremost a capacity for hearing sources of sounds does not mean that we cannot hear sounds themselves. Sound, as we saw, can be regarded a primary object of audition. When we hear, we always hear sound, because it is the perception of a sound that enables perception of its source. 

Notice that we can think of the way hearing sound enables us to hear sources in one of several ways. Some may think that hearing the sound of its engines offers us perception of an Airbus in a way analogous to how our seeing a curtain may enable us to see someone who is hiding behind it. Yet it is equally open for someone to think that hearing the sound of the engine enables one to hear the aeroplane in a way analogous to how a looking in a mirror may enable us to see someone who is standing behind us. These would be very different glosses of perceptual enabling. The perception of the person behind the curtain could be classified as perception of an \emph{indirect} kind, a kind in which we perceive one thing by perceiving another; in contrast, the perception of the person in a mirror could still plausibly be regarded a \emph{direct} kind of perception: looking in the mirror enables one to see the person while it would be a mistake to think that we see them by, or `in virtue of', seeing the mirror. 

Setting these details about hearing sources aside, the general conclusion Nudds' argument allows us to draw is that, in some way or other,  our capacity to hear sound can only be understood as a capacity to hear sources of sounds. Hearing is a capacity for hearing sound sources. It is a capacity for hearing a variety of spatiotemporal objects and events that make themselves heard by producing sounds our sensory capacities are sensitive to. 

This is an important observation to make. For, if it is correct, then we can say that the austere, Berkelean view that we only ever hear sounds and never hear sources must affirm that our perceptual capacity to hear could not be exercised properly, which, making some plausible assumptions about what perceptual capacities are, is absurd. It is absurd to think that we can do something we can not do. Hence, we must suppose that one can hear sources of sounds.
%END SECTION (end)

% Sources cannot be constituents of auditory perception. Argument for premise (3) (fold)
\sect   The naive realist defends a metaphysical claim about perceptual experience. Strong naive realism maintains that if we can perceive something, then what we perceive must be a possible constituent of perception. Naive realism about auditory perception of sound sources will maintain, more specifically, that if we can hear sound sources, then those sources must be possible constituents of auditory perception. But are sources of sounds possible constituents of auditory perception?

Answering this question requires us to say something about the nature of constituents of auditory perception and the nature of a source of sound.  It can be shown that constituents of auditory perception necessarily have some features that sources of sounds necessarily lack. 

It should be uncontroversial to assume that it is because of our sensory capacities that we are sensitive to some particulars in our proximal or distal environment and not others. When we peer out the window, we are visually sensitive to the presence of the brownish and bulgy scare-crow in the field before us, yet we are prone to remain blind to the lingering ghosts of our ancestors, if there would be any. This is because the scare-crow possesses qualities that the ghosts, being the way ghosts are, lack: chromatic and achromatic colour qualities the sense of sight makes us sensitive to. 

Similarly, you are able to taste sugar, while some other food additives cannot be tasted, and some sounds, such as that emitted by some nineteen-seventies remote controls, cannot be heard. That they cannot is precisely because the additive and the high-frequency sound lack the kinds of sensible qualities our sense of taste and hearing respectively are sensitive to. 

And so we may say that an object can only become a constituent of perceptual experience if it possesses some of a range of qualities our sensory capacities are sensitive to. In case of visual experience,  we can see objects that are coloured or blazing with light. A loose definition of colour could be given to give a \emph{prima facie} delineation of the range of qualities which all visible objects share \autocite[p. 334]{martin2012aa}. 

To return once again to an earlier point, what is interesting about this range of visible qualities is that there is no special kind of object that has to instantiate them. Just as a tree's leaves can be green, so can a flash, a laser beam or a glow. A rainbow can be coloured just as a concrete wall, even though these objects differ significantly in fundamental properties. In visual perception, there is no one kind of individual that must be present in every exercise of one's power to see. 

The important structural difference between vision and audition, however, makes it that in auditory perception there is such a `primary' object of sense. There is only a single kind of thing that instantiates the sensible qualities required for hearing. As Urmson writes, 

\begin{quote}
As everything that we see, including glows and flashes, has some colour, though usually a pretty nondescript one, so every sound we hear has some timbre, though usually a pretty nondescript one and components varying in pitch; as we can list colours---red, blue, green, etc., so we can list timbres---acidity, reediness, brassiness, etc. \autocite[p. 126]{urmson1968aa}
\end{quote}

A jet engine will be coloured, but it cannot categorically be said to be low-pitched or loud. Those predicates only apply categorically to the sound the engine produces. It is only sound that can instantiate the range of sensible qualities---such as timbre, pitch and loudness---the sense of hearing is sensitive to. 

This suggests that only sounds can be constituents of auditory experience. Hence, sound sources could only be constituents of this kind if they we themselves sounds. Yet, it should be obvious that sound sources are not. A sound could never produce another sound, something that would have to be the case if the sound would count as a sound source. It would be absurd to say of a high-pitched squeak that it went without sound, or ask whether the sound of the neighbour's dog made a noise.

Berkeley reasoned that, because of the way only sound can possess the sensible qualities the sense of hearing is sensitive to, it is only sound we hear. In the previous section we saw how that could not be right. But we can see that Berkeley's reasoning may gets things partially right: because of the way one sound can possess the sensible qualities the sense of hearing is sensitive to it is only sounds that could be constituents of auditory experience. If this revived version of Berkeley's argument is sound, sources of sounds are not possible constituents of auditory perception. 


% 
% The second line would focus on the nature of audition as essentially temporal. Even if hearing would be sensitive to more qualities so sources could become constituents by that criterion, there is still room for dissent. For, what are the conditions for shaping in audition? Auditory perception is shaped essentially by change, by something's unfolding over time. Yet, sources of sounds, if they are objects, do not unfold. The condition for shaping in vision are spatial: it is because the constituent occupies space that it is a constituent. It has to occupy this space in a way that sounds do not: sounds can fill a space, whereas objects are normally merely contained. Analogously, but differently, objects do not occupy time whereas sounds do. 
% 
% Also this argument is ancient. Sight works `instantaneously, whereas hearing requires motion and therefore occurs in time.'(Frangenberg: 72)
% 
% Note: music of the spheres is too loud :Burnett:49
% 
% %END SECTION (end)
%  
%  
%  
% Some may think there is room for the naive realist to adopt a more permissive understanding the conditions of perceptual constituency. Instead of thinking that objects are eligible to cast their perceptual vote by virtue of their intrinsic sensible qualities, it could be suggested that it is ultimately the role such objects play in our interaction with the world. For, having perceptual qualities will ultimately not be sufficient for constituency. An object may be painted garishly, yet be stored in an entirely dark room, or cast in concrete. Such an object could not become a constituent of perceptual experience, irrespective of its having the sensible qualities it has. 
% 
% It would be a mistake that action potential is a necessary condition. It can at best be sufficient. But that it can be sufficient will in turn prove, it seems, that having the right kind of qualities is not necessary, but also merely sufficient. This is an important objection.
% 
% Or is it? Perhaps I should only address this objection at a later stage. 
% 
% In response, 
% 
%  it is action potential that makes for constituency. Constituents of one's experience are those particulars that offer us potential for interaction in a larger encounter with the world.But this is implausible. We surely are able to perceive a distant planet or star without perceiving anything else, and so that distant planet, that star, must count as a constituent of our experience. And yet, it would stretch the idea of action potential beyond informativeness if we would say that the distant planet, the star, is something that offers us potential for interaction in a larger encounter with the world. 
% 
% 
% 
% 
% 
% 
% 
% Imagine O making noise. The noise is a sound. O cannot shape one's experience unconditional, because the way O shapes one's experience will depend on what happens with the noise O makes. If the noise is affected by the insulation of the library, then the noise O produced may simply be too faint to be heard. And so O becomes inaudible in the library. For someone in the library O could not determine their experience, no matter what it would try. (so to say.)
% 
% Distorting influence of the medium. This makes it that sources cannot be constituents. 
% 
% 
% 
% But some who are skeptical of global representationalism may equally look askance at the strong naive realist. It could be suggested that, in actuality, perceptual experience is never that straightforward. They may take it to be at least plausible that various other factors, both perceptual and non-perceptual, impact on the overall character of experience. Atmospheric distortions, poor eyesight and the kind of affective gilding and staining of the kind Hume described can all equally be expected to play their part in determining the character of your experience at any given time. Perhaps this should move us to accept a weaker 
% 
% 
% Nonetheless, 
% 
% 
% \sect When he aims to characterise the essence of sense perception, Martin Heidegger speaks of reality's showing itself to us. His metaphor captures nicely the kind of direct confrontation with the world envisaged by the naive realist – a kind of worldly exhibitionism.

%Bibliography: \standardbib just loads a regular bibliography. All files have been loaded in the preamble.
\printbibliography
\end{document}

