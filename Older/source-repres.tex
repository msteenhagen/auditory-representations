% 
%  thoughtsnotthought.tex
%  
%  Created by Maarten Steenhagen on 2013-07-01.
%  Copyright 2013 Maarten Steenhagen. All rights reserved.
% 
\documentclass[sloppy, journal, git, anonymise, dodraft]{humapap}
\usepackage{soul}
\myauthor{Maarten Steenhagen}
\myemail{\\m.steenhagen.09@ucl.ac.uk\\} % Will be used on first page, loaded in author.tex
\mydraft{draft version}
\theaffiliation{}
\thetitle {source representationalism} % Fist page (no caps is fine)
%\mysubtitle{a second title} % First page
%\mydraft{Draft - do not cite}
\thehtitle{the header} % Title in the header (no caps is fine)
\mydescription{draft version} % Hovers over the title. Can be left blank (~)
\thejournal{the journal} % Left corner subsequent pages
\theyear{2014}
\thanks{}
\remote{https://bitbucket.org/msteenhagen/humapap/commits/}
\begin{document}
\documenttitle

\begin{abstract}
\dropcap{S}{ource representationalism} is the thesis that the sources of the sounds humans hear, whenever they are heard, are represented in experience. I defend this thesis on the basis of considerations about listening to sound recordings. I argue that listening to a recording and to an actually produced sound could be indistinguishable, while in both cases still a sufficient condition for perceptual representation is met. I conclude that this should lead us to accept source representationalism as a global thesis about audition. 
\end{abstract}	

% INTRODUCTION (fold)
\dropcap{S}{ource representationalism} is a view about auditory perception,
according to which the sources of the sounds we hear, whenever they are
heard, are represented in our perceptual experience. When we hear a
piano being played, the source representationalist may accept that our
auditory experience consists in a perceptual relation to the sound of a
piano, yet they will deny that it consists in a perceptual relation to
that piano itself. In this paper I defend source representationalism on
the basis of some considerations about listening to audio recordings.



\sect Consider the situation in which you listen to a stereo recording of
Keith Jarrett's solo piano improvisations---\emph{The Köln
Concert}---performed on the 24th of January 1975 at the Cologne Opera
House. In contrast to its actual audience, which found itself in a
celebrated concert hall designed by Wilhelm Riphahn, you are currently
seated in London, in a small and quiet room with only a chair, writing
desk and a high fidelity speaker set in your vicinity. You know that
both the quality of the recording and the calibration of the
loudspeakers in the room are superb, and sit back at ease, eyes closed,
well aware that you are in for a treat. Carefully you press the button
that will start playback. Now what can you hear?



\sect First of all, you can hear the sound of a piano. This particular sound
becomes perceptible as soon as it fills the room when the recording
starts to play. On closer attention to Jarrett's performance, the sound
may begin to decompose into individual sounds of piano keys pressed
jazzily by a pair of skilled hands. Let us assume that all these sounds
you can hear are instances of the sound of a piano.

Contrast this experience with one in which a conceivable device is used
which, instead of playing back sounds, causes us to have auditory
hallucinations. Playing back a recording on such a device would result
your having a series of auditory sensations, sensations that in some
respects resemble tinnitus and in other, more subjectively salient
respects resemble hearing Jarrett's solo piano improvisations. Enjoying
a piano recording by means of such a device would not involve hearing
the sound of a piano, but instead work through auditory hallucinations
of hearing such a sound.

It should be clear that listening to an ordinary playback of a sound
recording of Jarrett's \emph{Köln Concert} work is no hallucination. It
is a perception of sound. Some may want to say, more specifically, that
it consists in a perceptual relation to sounds of a piano that trickle
into the room after the playback of the recording has started.

Someone might ask: What could be meant by saying that here we hear the
sound of a piano? Can we really hear the sound of a piano in the absence
of any such instrument? I think we can. Some maintain that it is of the
nature of sounds that they are reproducible \autocite{martin2012aa}. If
this is right, then your speakers could reproduce the particular sounds
that were produced by a piano on the 24th of January 1975.

Some may wish to resist considering sounds to be reproducible and
abstract in this way. Yet even they can agree that in listening to the
playback one hears some particular sounds that, given a well-calibrated
stereo set, are qualitatively identical to---and hence indiscriminable
from---the sounds that an actual piano could produce. I would like to
invite those reluctant to accept that sounds are reproducible to keep in
mind that, wherever the argument that is to follow relies on our hearing
the sound of a piano, it is at any time possible to understand this as
committing to no more than the claim that we hear a sound that, at least
ideally, is qualitatively identical to the sound an actually played
piano could have produced.



\sect In general, the claim that we can hear more than sounds alone is by many
authors taken to be undeniable. J. O. Urmson writes,

\begin{quote}
Clearly one may see, hear, feel, smell, and taste physical objects like
motor cars and apples; it requires ingenious stage setting to make `I
hear an apple' or `I taste a motor car' sound natural \ldots{} but one
may certainly hear a motor car or taste an
apple.\autocite[p. 117]{urmson1968aa}
\end{quote}

Just as we can see a magpie, or taste the vinegar in our salad, we can
hear things that are not sounds: people in the corridor, aeroplanes
overhead, a pub fight down the street. There simply is no reason to
adopt the restrictive view, advanced by Berkeley's mouthpiece Philonous,
that in general sounds are all we hear \autocite{berkeley1954aa}.

Notice that, although it certainly is one, this observation is not
merely a report of common sense; it has a compelling phenomenological
basis as well. Both sounds and sources can be made object of attention,
and on occasions even compete for it. An increasing number of authors
emphasise that the possibility to attend to things that are not sounds
reveals that the objects of audition include more than sounds alone
\autocite[See:][]{ocallaghan2009aa}.

Imagine that during the night you hear a sound coming from your living
room. You realise that it's the cat who managed to open the kitchen door
again. Reluctant to get out of bed, you first listen more carefully to
what goes on. You'll not pay much heed to the sounds you hear. Of course
you could try to set yourself to attend to them, but you don't, because
what your attention will go out to is what matters to you at that time:
a feline threat to designer furniture. You attentively listen to the
cat; you may listen to it scratching the sofa, or, if you're unlucky,
hear it throwing over a vase (and, inevitably, hear the vase breaking).

In hearing a sound, we can often attend to, and hence listen to, what
the sound is of. \emph{Pace} Berkeley, not only would we be very happy
to characterise the experience described above as one of hearing a cat,
the fact that we can also distinguish between those two acts of
attention, proves that we can hear more than sounds alone.

This means more than that we are able to hear that they are the sounds
of a cat, or that the cat is in the room. Such `epistemic' hearing of
facts about sounds or pets does of course occur, but the point is
different: it is that attentively listening to the sound and to the cat
only seems possible if both sound and cat can be heard. Just as Urmon's
apple, your cat, as the source of the sound you hear, can be an object
of audition.

If this is right, then by the same reasoning we may conclude that in
listening to a recording we can hear a sound's source. Consider an
example. To find out what had happened in the Boeing 737 that crashed
near Pittsburgh in September 1994, a team of audio forensic
investigators played back the cockpit voice recorder that was retrieved
after the accident. The team was particularly interested in a
malfunctioning of the plane rudders, and it is unlikely that they only
attended to the sounds reproduced by their playback equipment. Indeed,
they found out that rudder's eventual jamming was not due to mechanical
obstruction, but because of inept operation of controls. Such a finding
was only possible through listening attentively to two sources that were
audible in the recording: the plane's rudder and the crew's manual
operation of the dashboard's switches and dials. Hence, also in listening to a
recording it is possible to turn one's attention to a sound's source,
and hear it.

Whether one can engage in this mode of attentive listening to sources turns on whether one hears the right sounds, not on whether it is a
recording. However, someone might object that the case of hearing
a cat at night and hearing a plane's engine in a recording are
different. Where in the former example there actually is a cat producing
those sounds, what is directly responsible for any recorded sound one
hears is a mere loudspeaker. Hence, the objection would run, the fact
that one can attend to something beyond the mere sound of a recording
suggests just that we can hear the loudspeaker that reproduces it.

This objection is unconvincing. As every audiophile will
attest, loudspeakers are inaudible if they function well. Understanding
what it would be to hear a loudspeaker requires consideration of cases
where the device has a noticeable defect. Imagine that, while listening to the recording of Jarett's play, an annoying
tearing noise makes itself heard. Your loudspeaker is bust, and its sound
now interferes with the piano's melody; such a tearing will stand
out in experience as not belonging to the recording itself. This is what
hearing a loudspeaker is like. It should be clear that, instead of
setting a standard for listening to recorded sounds, such an experience
just signals that equipment needs to be replaced.

\sect The following is a paradigmatic sufficient condition for perceptual
representation:
\begin{description}
\item[Perceptual representation] If a perceiver perceives some object that is not present to the senses, then the object is represented in perception.
\end{description}
This enables a grasp on the phenomenon of perceptual representation,
because the condition allows us to identify typical occurrences. In
particular, it allows us to say that if we merely play back a recording
of a piano, then any hearing of a piano counts as an instance of
perceptual representation. This is because we would hear the instrument
while it is not present to the senses. Hence, hearing the sound of a
piano suffices for being able to hear a piano representationally.

Assuming that there is nothing peculiar about pianos, we should accept a
weak version of source representationalism. Whenever we hear a source
while merely playing back a sound recording, the source is represented
in experience because we hear it while it is not present to the senses.

This raises a question. By what mechanism is perceptual representation
of this kind possible? Merely suggesting we possess a capacity for
representational perception may not be entirely satisfactory. Yet, we
can say more. No doubt the heard sound plays a determining role in the
exercise of this capacity. Just as other perceptual capacities, the
capacity for perceptual representation is only exercised in response to
something---it is a \emph{reactive} one
\autocite[cf.][]{kalderon2014aa}. This suggests that a perceived sound
figures as an essential cog in the machinery of representation in
auditory perception: by determining in part the character of our
experience, it is the heard sound that will represent a piano to us.
Naturally, this leads us to ask what it is about that sound, as an
auditory object, that lets it take on such a representative attire.

Here I do not want to address this further. Whatever account we give of
the nature of sounds, or of perceptual representations more generally,
if the previous discussion is along the right lines, we do possess a
psychological capacity to make use of them. Hearing sounds with the right character suffices for an exercise of this capacity. In what follows I want to
show that this implies that source representationalism should be accepted as a stronger, global thesis about auditory perception.

\sect We saw that we can hear both the sound of a piano and a piano in
listening to a recording of Jarrett's \emph{Köln Concert}. We reasoned
that in such circumstances, although the sound may be present to the
mind, the piano we hear must be represented in auditory
experience---the circumstances enabled us to hear such an instrument in
its absence. Now is this representational dimension also an aspect of
perceptual experience when an instrument is actually played to us?

We should begin by noting that listening to a recording may be
perceptually indistinguishable for us from listening to an instrument
being played in real life, and vice versa. Think of those talented
musicians in the bowels of London's Underground. At some distance, one
may be in doubt whether they are actually strumming their guitar or are
relying on a covertly playing recording. When in doubt about this,
reflecting on one's auditory experience may not be sufficient to
distinguish between the two possible scenarios, since for any actual
performance, there is a conceivable, indiscriminable counterpart that
makes use of a recording.

This is a general point about audition. For every animal, aeroplane or
midnight brawl one hears, it is always conceivable that one has such an
experience because one hears sounds reproduced via a recording.

What makes hearing an actual fight outside one's bedroom window and a
recorded one potentially indistinguishable? This question has a clear
answer: both consist in an straightforward perceptual awareness of the
sound of a brawl. Hearing a brawl via a recording and hearing a brawl
because a pub fight takes place just below one's half-opened window can
be indistinguishable because in both cases one could be perceptually
aware of the exact same sound, a sound that in each case could have the
same impact on the character of one's experience.

Our earlier discussion suggested that one's hearing the sound of a piano
is a sufficient condition for one's being able to hear a source
representationally. And if it is a sufficient condition for this ability
to become exercised, then also when we hear a sound when it is actually
produced by a fighting mob or by a concert piano in 1975 are we in a position to
hear its source representationally. Given the above observations about
the conceivable indistinguishability of recorded and real-life cases,
this conclusion should come as no surprise.

Hearing a piano via a recording and hearing a piano while listening to
an actual performance are both cases of perception. It may be said that
in both cases we are perceptually related to the same thing: the sound
of a piano. In both cases, it is the sound of the piano that in part
determines the character of our auditory experience. Moreover, we know
that just on the basis of having an experience with such a character, it
may be impossible to tell just from what we hear whether we are hearing
a piano that is actually present or a piano that is merely recorded.
This suggests that the actuality of a piano in our surroundings makes no
contribution to the character or nature of auditory phenomenology. If this is right, then whenever we hear a piano we do so in a representational way.

\sect As it is defined here, source representationalism is the view according
to which the sources of the sounds we hear are, whenever we hear them,
represented in experience. Reflecting on listening to things through
recordings, and on the conceivable indistinguishability of hearing
recorded sounds and hearing sounds that are actually produced, has
opened up a compelling line of defence of the source representationalist
claim. Whenever we hear a piano or some other sound source, this object
of audition is represented in experience.


\sect The argument developed here does not directly impact on the debate
between naive realists and representationalists about perception. Source
representationalism remains silent on whether all perceptual experience
is representational. This should be emphasised.

The representationalist about perception maintains that all perceptual
experience is representational. Perceiving \emph{x} just is perceiving
\emph{x} representationally; whatever it is we see, hear or feel, it is
represented in experience when we see, hear or feel it. A naïve realist,
on the other hand, denies this. They maintain that not all perceiving is
representational.

The source representationalist defends that at least some aspects of
perceptual experience are representational whenever they occur. Hence, a disagreement with
the representationalists about perception could only arise if some
additional assumption is made. For instance, if the source
representationalist in addition assumes that perception of a sound can
consist in a perceptual relation, something that could only obtain in
the actual presence of the sound, they would be siding with the naïve
realist. But making such an additional assumption is optional. All the source representationalist thesis suggests is that a complete picture of the
scope of perceptual experience must acknowledge the role perceptual representations can play.

%Bibliography: \standardbib just loads a regular bibliography. All files have been loaded in the preamble.
\printbibliography
\end{document}

