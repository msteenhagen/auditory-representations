
\documentclass[sloppy, journal, bytitle, dodraft]{humapap}
\usepackage{soul}
\myauthor{Maarten Steenhagen}
\myemail{\\m.steenhagen.09@ucl.ac.uk} % Will be used on first page, loaded in author.tex
\mydraft{Draft version}
\theaffiliation{}
\thetitle {following audition's lead} % Fist page (no caps is fine)
%\mysubtitle{a second title} % First page
%\mydraft{Draft - do not cite}
\thehtitle{the header} % Title in the header (no caps is fine)
\mydescription{Draft version} % Hovers over the title. Can be left blank (~)
\thejournal{the journal} % Left corner subsequent pages
\theyear{2013}
\thanks{~}
\remote{https://bitbucket.org/msteenhagen/humapap/commits/}
\begin{document}
\documenttitle

% This is the abstract submitted to the conference, and it will probably be distributed. 
\begin{abstract}  Ordinary seeing is not representational, but some seeing is. In this paper I identify and explain a structural difference between audition and vision, and use this to show how auditory perception is ordinarily a kind of  representational perception. I suggest that reflection on how audition can be representational  sheds light on the nature of representational seeing. \end{abstract}

\dropcap{T}{he conditions} under which visual perception could be representational are not well understood. In what follows I begin by making some observations about the role sensory capacities have in perception of particulars. I use these to bring out a difference between visual and auditory perception. I suggest that, because an exercise of our sensory capacity of sight can relate us to a mixed bag of middle-sized particulars in our surroundings, ordinary vision does not involve perceptual representation. Yet, I suggest that reflection on equally ordinary cases of auditory perception, which, I argue, do give us reason to consider ordinary audition to be representational, helps us understand how, on specific occasions, also visual perception can become representational. My ultimate aim is to show how reflection on the nature of auditory perception can bring into view a notion of visual representation that is not easily revealed by an exclusive focus on vision as such.

\sect We perceive objects by becoming aware of their sensible aspects or qualities. The stalk of rhubarb I clench in my hand, here in my well-lit studio, instantiates sensible qualities of familiar kinds: it is a coloured stalk. Colour is the distinctive sensible quality our capacity of sight is sensitive to. And so it is because my sensory capacity for sight makes me sensitive to this distinctive kind of sensible quality that I can come to see the particular plant I hold and manipulate. 

That, in a nutshell, is vision. 

But reflection on the example of seeing the stalk of rhubarb I am handling, and on the role my sensory capacities have in it, suggests a plausible way of conceiving of all situations in which we can be said to exercise those capacities in perceiving particular objects.
 
By exercising any of our sensory capacities we can perceive particulars in some perceptual modality. That is what sensory capacities are for. Perception of some object takes place by virtue of a particular object's instantiating the specific kinds of sensible qualities specific sensory capacities are sensitive to. But sensory capacities come in different flavours. Whereas sight is sensitive to the plant's greenish redness, the sense of touch by contrast is sensitive to the plant's hardness and coldness, and our sense of taste picks up on yet different qualities, such as the rhubarb's distinctive sourness. 

Now, one might press that this may not cover all ways we can perceive objects. That would be correct. Sometimes, we can perceive one object by becoming aware of the sensible qualities of another---for instance, when we see a rising column of hot air by becoming confusedly aware of the sensible qualities of the scene we see through it. Given this, it seems that our perceptual capacities can be used inventively, in ways that go beyond the general way sketched in the previous paragraph. 

It would be fair to press this point. But it is important to see that even if one accepts perception's inventiveness, perceiving an object by becoming aware of its sensible qualities may still be taken to be the most fundamental way---it is the way that underpins in one way or another any perceptual situation, no matter how inventive it is. 


\sect The previous offers us a plausible picture of what perception is, fundamentally. If we conceive of the perceptual situation as fundamentally a situation in which an exercise of our sensory capacities, by affording us awareness of instances of sensible qualities, allows us to perceive particulars that instantiate those qualities, then we can explain a striking difference between vision and audition. How does this explanation run?   

The kind of difference between vision and audition I have in mind has often been commented upon, and it is of considerable philosophical importance. M.G.F. Martin, for instance, characterises it in the following way: 

\begin{quote}
What puts vision apart from hearing \ldots is that we do not conceive of the visible world as offering us primary objects of visual awareness and attention distinct from concrete objects through which we come to see the concrete.\autocite[p. 334]{martin2012aa}
\end{quote}
Martin rightly observes that an exercise of our capacity for sight is generally not regarded as bringing us into contact with some intermediary object distinct from the mixed-bag of concrete particulars that make up ordinary reality. Yet, in audition we do seem to presuppose a kind of primary object of perception that is distinct from the concrete and familiar world around us. We take it that hearing is always a hearing of sounds. 

The earlier conception of the perceptual situation  allows us to identify the features of both kinds of perceptual situation that are responsible for this difference between vision and audition. These features are the sensible qualities our sensory capacities are sensitive to. Colour qualities are categorical qualities of various kinds of particulars such as stalks of rhubarb, light rays, pigs and fireballs. But the kinds of sensible qualities our sensory capacity for hearing is sensitive to---pitch, timbre and the like---are only ever categorical qualities of a single kind of audible particular. That is, they are only ever the qualities of sounds. It is the sound of the piccolo that is high-pitched. It is the cello's sound that instantiates the mellow timbre we commonly associate with this specific kind of string instrument. 

If this is right, then it suggests a straightforward way of explaining the contrast between vision and audition. Just by virtue of an exercise of our sensory capacity of sight can we be perceptually related to a variety of material and immaterial objects, such as pigs and rainbows and stalks of rhubarb. This is because this capacity makes us sensitive to a specific range of sensible qualities of these particulars---by making us sensitive to colour instances, the sense of sight in this way puts us in touch with the everyday world. Yet, in contrast, an exercise of our auditory capacities on their own can only deliver us something more limited. Precisely because our sensory capacity to hear makes us sensitive only to qualities of sound, by merely exercising our sensory capacity to hear can we only become aware of the sounds that reach our ears. The sense of hearing puts us in touch with only that aspect of reality that is purely auditory, with a world comprised of sounds alone. 

It is crucial to stress that the previous does not imply that we only ever hear sounds. Although Berkeley and others do seem to regard this to be our condition, I believe we should take seriously the fact that we ordinarily think of audition as a perceptual capacity that makes us aware of a variety of objects and events in everyday life. Reflection on the concept of auditory perception suggests that we hear more than sounds alone. As C.D. Broad observes, 
\begin{quote}
It is about equally common to speak of hearing a \emph{body} and of hearing a \emph{sound}. Thus, e.g., one can say: ``I hear Big Ben" and ``I hear a series of booming noises."\autocite[p.4]{broad1952aa}
\end{quote}
If one takes Broad's observation seriously, one is likely to be sympathetic to the idea that, \emph{pace} Berkeley, besides the sounds that reach our ears, we can also come to hear the particulars that produce these sounds, such as Big Ben, or the events of such sounds' production, such as Big Ben's striking at 6pm, or both. 

The point of emphasis, then, is that the earlier claim  that our sensory capacities to hear make us sensitive only to sounds is compatible with the more liberal assumption about what we can come to hear. We can assume that we can hear Big Ben or that we can hear Big Ben's striking and still accept the earlier claim about sensory capacities. All that is implied by that claim is that our capacity to hear the producers and the productions of sounds cannot be explained merely in terms of the exercise of our sensory capacity for hearing. This is because our sensory capacity for hearing is just not sensitive to any of the sensible qualities events of a sound's production or a sound's producer may possess. But there is no good reason to assume that our capacity to perceive some particular object or event must be explained exclusively by the exercise of the various sensory capacities we have. 

\sect In what follows, I will assume that the earlier picture of what perception is, fundamentally, is correct. I argued that if we conceive of the perceptual situation as fundamentally a situation in which an exercise of our sensory capacities, by affording us awareness of instances of sensible qualities, allows us to perceive particulars that instantiate those qualities, then we can explain a striking difference between vision and audition. We have seen how this explanation runs, but it has left us with a further question. If we assume that we can hear more than sounds alone, then how is that possible?

I want to suggest that this points to a more general question about representation in perception. We can define representational perception in the following way. Perception in some modality is representational at some moment if and only if the particular objects we are at that moment able to perceive by exercising the sensory capacities that belong to that modality do not exhaust the particular objects we can perceive in that modality at that moment.  

To return to the auditory case, we can see that ordinary audition takes this kind of representational form. By exercising our sensory capacities for hearing we come to hear the sound of Big Ben, yet we have also assumed that we can hear more than that sound alone---we can hear Big Ben itself, or the swinging back-and-forth that produces the sound. In that way, the sound we are able to perceive by exercising our sensory capacities for hearing does not exhaust the particulars audition allows us to perceive. Auditory perception, if it takes this form, is a kind of representational perception. It is a kind of perception that bears a representational content, at least on the occasions where we hear more than what an exercise of our sensory capacity to hear can explain.

As a plausible hypothesis, what now suggests itself is that representational hearing is possible because the exercise of our sensory capacity to hear is supplemented by an exercise of a cognitive capacity to recognise sounds as the sound of something or other. It is because we can hear the sound of Big Ben as the sound of Big Ben striking that we can hear Big Ben.

\ldots incomplete \ldots


\printbibliography
\end{document}